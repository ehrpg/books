\documentclass[11pt,a4paper,openany]{scrbook}
\usepackage[usenames,dvipsnames]{xcolor}
\usepackage{color}
\usepackage[utf8]{inputenc}
\usepackage[T1]{fontenc}
\usepackage{multicol}
\usepackage{hyperref}
\usepackage{mdframed}
\usepackage{verbatimbox}

\begin{document}

\title{Endless Horizons}
\author{Johannes Zwirchmayr, Florian Stinglmayr}
\maketitle

\chapter{Licence}

  This document is licenced under Creative Commons
  \href{http://creativecommons.org/licenses/by-nc-sa/4.0/}
       {Attribution-NonCommercial-ShareAlike 4.0 International}.
  Which means that you may freely share and modify this document.
  But you may not use it commercially.

\chapter{Introduction}

\subsection{Endless Horizons}

Endless Horizons is a pen \& paper game set in a futuristic space setting. To
play you will need various dice and a sheet of paper.

\subsection{Game Master}

A game of Endless Horizons is lead by one person the so called Game Master.
He controls the flow of the story, sets up events and prepares the
session ahead of time.

\chapter{Character Creation}

The following sections will guide you through character creation. Characters
in Endless Horizons are your average Jacks and Jills that fight for survival
in the ultimate hostile space: the known universe.

\subsection{Character Creation Points - CCP}

The entire process of creating and advancing your character is driven by so
called \emph{Character Creation Points} or \emph{CCP} for short. You receive
a given amount at the beginning to build your character, and receive more as
you progress in the game.

\emph{CCP} allow you buy ability increases, improve skills, and even allow you
get special features for your character called perks.

Most games will have you start with 50 \emph{CCP} at the beginnign for character
creation. But it is always up the GM to fiddle with this number so that it fits
his campaign.

\subsection{First: Abilities}

Characters are defined by six major abilities. An ability can have a rank from
one to ten, with the average being five. Each ability also has a modifier. This
modifier represents if you gain a benefit or a disadvantage from your ability
rank. For each rank above five, you gain one modifier, and for each rank below
five you gain one minus modifier. So if your rank is three, you have a minus
two modifier, and if you rank is seven you have a plus two modifier.

The abilities are the following:

\begin{itemize}
\item \emph{Strength} - Strength represents your raw physical power. It
  determines not only how much a character can lift, drag or push. But also how
  far he can jump, or how much damage he does in melee combat.
\item \emph{Dexterity} - Determines your characters fine motor skills and how
  well your hand-eye coordinaton it is.
\item \emph{Constitution} - This ability determines how physically well your
  character is. It determines hit points, as well as endurance and overall
  fitness of your character.
\item \emph{Intelligence} - Intelligence represents how well a character learns,
  how well he can thing logically, or in abstract terms.
\item \emph{Perception} - Perception is the awareness of the character. It
  describes how well a character can describe his surroundings, either passively
  or actively. It determines how well he finds things that are hidden or
  well he notices small occurences around him.
\item \emph{Charisma} - Charisma determines how well a character can interact
  with other characters. This includes things as speech, facial expressions,
  gestures, articulation and expression.
\end{itemize}

Your character starts with \emph{five} (5) in all abilities. At the beginning
you may buy or sell ability ranks with CCP. Buying the next rank costs exactly
as many \emph{CCP} as the rank. So an increase from five to six would cost six
\emph{CCP}.

The reverse is also true if you lower an ability: You receive as much \emph{CCP}
by lowering as it would cost to raise the ability to the old value. So from
lowering an ability from five to four, you'd receive five \emph{CCP}.

At the beginning you may raise an ability not higher than seven, and may not
lower it beneath three. Later on in your adventuring career the GM may allow you
to increase an ability further. But that is up to the GM's discretion.

\subsection{Second: Pick a Background}

The game has your character learn skills to be useful. Backgrounds are a set of
predefined set of skills that fit to a certain theme, or job a character might
have learned throughout his career. They are optional, however, and experienced
players might wish to pick the skills for their character individually.

Backgrounds should help you to get you started with creating your character, but
not limit you in the possibilites. If you feel that none of the background fits,
and you do not wish to pick the skills on your own, talk to the GM.

A background usually comes with an ability increase. This increase stacks with
whatever you picked in the first step. A background also comes with several
skills unlocked and at certain ranks.

You are free to select more than one background, but neither the skills nor the
ability increases stack. Only the highest one applies.

\subsection{Third: Pick Skills}

Picking skills is the most important aspect of \emph{Endless Horizons}. There
are two types of skills: Special skills and normal skills. Special skills take
longer than normal skills to learn, or even require access to special equipment
or training facilities.

Each skill your character wishes to learn has to be unlocked. Unlocking costs
one, for normal skills, and ten \emph{CCP}s for special skills. Skill ranks
start at zero and go until ten. Each increment costs exactly as many
\emph{CCP}s the increment you wish to buy.

\chapter{Checks and Difficulties}

\subsection{Overview}

Everything in this game is done by making a \emph{check} against a given
\emph{difficulty}. It is the players job to make the check, while it is the
game masters job to keep track of the difficulty. When bonuses or penalties
are given in this document, they apply to the difficulty \emph{and not the
  check}.

This is important as the player probably does not, and should not, know all
the factors that make his work harder. As a game master you should always be
be as transparent as possible when it comes to determining the difficulty.

The game does not give the GM a detailed list of specific bonuses or penalties,
and it is up the GM's discretion to specify this. As an example, a fog can give
add from +1 to +6 to the penalty of a check to identify another person,
depending on how thick it is.

\subsection{Rolling}

Almost all of the skill checks in \emph{Endless Horizons} are done by rolling
a ten sided die (d10). The player adds any relevent bonus from his or her
character, while the game master calculates the difficulty. If the player's
result is at or above the difficulty he succeeds.

If the player rolls a one (1) or a ten (10) he has caused the dice to escalate.
The player rolls again until either he rolls a number that is not one or ten,
or until he has rolled three times in total. Rolling a ten gives a minor
positive effect, rolling a one adds minor negative effect. Rolling two tens or
two ones in a row adds a medium positive or negative effect; and rolling three
ones or tens in a row adds a major positive or negative effect.

Positive and negative effects cancel each other out. So if the player rolls a
one, and then rolls a ten; these two effects cancel each other. If the player
rolls a ten, then another ten and finally a one, the result is just minor
positive effect as one ``ten'' and the ``one'' cancel each other.

A check is not automatically failed when rolling a one. A check only fails if
the result does not meet the difficulty.

\subsection{Effects}

When the player achieves a minor, medium or major effect it is up the game
master to decide what happens. These effects happen regardless of whether the
check succeeds or not.

Minor positive effects grant a small immediate bonus (for example extra
damage, or the task is done faster), while minor negative effects do the
opposite. For example a task could take longer to complete, or the player
does less damage than normal.

Medium effects give major bonuses (for example they might take down the target
outright) or major disadvantages, for example the weapon jams and needs to be
cleared and reloaded before it is usable again.

Major effects are relatively rare, but may bring devastating results not only
for the player in question but maybe also for others around him, or allow him
to achieve the seemingly impossible.

Example: The player wishes to shoot a small burst with his rifle against a
target down the hallway. He rolls a 1, but after all the bonuses are added
the GM determines that he hits. The player rolls again (a 6) and breaks the
escalating die. He hits, although with a minor negative effect. The game
master decided that in the heat of battle the player overshot, and spent
more bullets than he intended: ten instead of five, and none of the extra
bullets hit any enemies.

\chapter{Skills}

\subsection{Overview}

When a character peforms complex operations a skill is involved, that your
character has to learn first. Whether it be operating a vehicle, shooting a
firearm, climbing a steep cliff or trying to convince another character or
NPC. Each skill is associated with two abilities. If a skill check is made,
a trained character can add the modifier of those abilities to his roll.

Some skills are special, as in they require a lot of effort to learn and use
right, or require special materials, trainers or facilities to learn properly.
There are also speciality skills. These skills cannot be learned on their own,
but reflect a specialisation of an already existing skill. They help to
specialise and refine a character.

Each skill has a rank that goes from zero to ten. With zero meaning that the
character has just learned the skill and cannot draw upon any experience when
performing tasks. Ten represents absolute mastery.

\subsection{Learning new skills}

Normal skills take one \emph{CCP} to unlock, upon which the skill has a rank
of zero. If the skill is a special skill, the unlock costs are ten \emph{CCP}
instead.

Once the skill is learned, rank increases can be bought with the same exact
number of \emph{CCP} as the next rank.

\subsection{Using Skills}

Each task, against which your skill is pitched against, has a base difficulty.
If the base difficulty is at or below your skill level you may be able to
perform the task without rolling your dice in come circumstances. This means
that your character can perform this task given enough time and no adverse
external influences.

For example, an avid shooter can always hit his target on his range, if he's
rested and aims long enough. But this doesn't mean he can hit another target
on the same distance as easy in the heat of combat, fatigued by days of
fighting. As soon as external influences come into affect, you \emph{have to
  roll}.

If you have unlocked a skill you make a skill check rolling a ten-sided dice,
adding the appropriate ability modifiers, adding the skill rank and adding any
specialised skill ranks (if applicable).

If you have not unlocked the skill you simply roll a ten-sided dice and only
add negative ability modifiers to the roll.

When your result exceeds or is equal to the tasks difficulty (which is the base
difficulty, plus any situational hindrances) you succeed. Please see tasks and
\emph{Checks and Difficulties} for rules on causing minor, medium or major
successes or failures.

\subsection{Specialities}

Specialiaties are specialised sub-fields of another skill. They represent
niches, or specialised area your character can focus on. You are allowed to
take up to three specialised skills (if there are that many), depending on
the rank of the parent skill. Once you reach rank three on the parent skill
you make take one specialised skill, another at rank six and another at rank
nine.

Your specialised skill does not need unlocking, but you cannot have the rank
of the specialised skill above the rank of the base skill. If you perform a
skill check that involves that specialised skill you can add the rank of the
specialised skill to the skill check (alongside the skill rank of the base
skill).

For example, Hana a skilled physician wishes to specialise in Surgery. After
reaching Medicine at rank three, she buys two ranks of Surgery. Now when she
has to perform a surgery she can add the three from Medicine, and the two from
Surgery to the skill check for a total of plus five.

\subsection{Available Skills}

Below you will find a list of all the base skills available in game. It is of
course up the GM to adapt or add to the list below.

%% TODO: Sorting!

\linebreak
\emph{Melee Combat}: You are trained in the art of hand to hand combat. This
skill includes basic training with common melee weapons (knife, batons or
clubs). To hit an opponent you use \emph{Dexterity} and \emph{Strength}.
The damage depends on the weapon used.

\linebreak
\emph{Small Arms}: The small arms skill allows a character to aim, shoot, reload
and maintain certain fire arms. Many firearms that are can be learned without
special training, while some others (especially military grade hardware)
requires special training (see \emph{Light Weapons}). Firearms rely on
\emph{Perception} and \emph{Dexterity}. Rifles and handguns handle differently
and are thus split up in two separate speciality skills.
\emph{Small Arms (Handguns) (Speciality)}: Handguns include firearms such as
pistols, revolvers and some specific sub machine guns. They have limited range
and stoping power but are easily concealed.
\emph{Small Arms (Rifles) (Speciality)}: This category includes sporting and
hunting rifles, as well as shotguns.

\linebreak
\emph{Light Weapons}: This skill grants training in military grade hardware,
such as recoilless rifles, grenade launchers, mortars, light and heavy machine
guns. It uses \emph{Perception} and \emph{Dexterity}.

\linebreak
\emph{Heavy Weapons}: Heavy weapons are all weapons that cannot be effectively
carried into combat, and include things such as a howitzers, large mortars or
turrets.

\end{document}
