\documentclass[11pt,a4paper,openany]{scrbook}
\usepackage[usenames,dvipsnames]{xcolor}
\usepackage{color}
\usepackage[utf8]{inputenc}
\usepackage[T1]{fontenc}
\usepackage{multicol}
\usepackage{hyperref}
\usepackage{mdframed}
\usepackage{verbatimbox}

\begin{document}

\title{Endless Horizons}
\author{Florian Stinglmayr}
\maketitle

\chapter{Licence}

  This document is licenced under Creative Commons
  \href{http://creativecommons.org/licenses/by-nc-sa/4.0/}
       {Attribution-NonCommercial-ShareAlike 4.0 International}.
  Which means that you may freely share and modify this document.
  But you may not use it commercially.

\chapter{Introduction}

\subsection{Endless Horizons}

Endless Horizons is a pen \& paper game set in a futuristic space setting. To
play you will need various dice and a sheet of paper.

\subsection{Game Master}

A game of Endless Horizons is lead by one person the so called Game Master.
He or she controls the flow of the story, sets up events and prepares the
session ahead of time.

\chapter{Character Creation}

The following sections will guide you through character creation. Characters
in Endless Horizons are your average Jacks and Jills that fight for survival
in the ultimate hostile space: the known universe.

\subsection{Character Creation Points - CCP}

The entire process of creating and advancing your character is driven by so
called \emph{Character Creation Points} or \emph{CCP} for short. You receive
a given amount at the beginning to build your character, and receive more as
you progress in the game.

\emph{CCP} allow you buy ability increases, improve skills, and even allow you
get special features for your character called perks.

Most games will have you start with 50 \emph{CCP} at the beginnign for character
creation. But it is always up the GM to fiddle with this number so that it fits
his campaign.

\subsection{First: Abilities}

Characters are defined by six major abilities. An ability can have a rank from
one to ten, with the average being five. Each ability also has a modifier. This
modifier represents if you gain a benefit or a disadvantage from your ability
rank. For each rank above five, you gain one modifier, and for each rank below
five you gain one minus modifier. So if your rank is three, you have a minus
two modifier, and if you rank is seven you have a plus two modifier.

The abilities are the following:

\begin{itemize}
\item \emph{Strength} - Strength represents your raw physical power. It
  determines not only how much a character can lift, drag or push. But also how
  far he can jump, or how much damage he does in melee combat.
\item \emph{Dexterity} - Determines your characters fine motor skills and how
  well your hand-eye coordinaton it is.
\item \emph{Constitution} - This ability determines how physically well your
  character is. It determines hit points, as well as endurance and overall
  fitness of your character.
\item \emph{Intelligence} - Intelligence represents how well a character learns,
  how well he can thing logically, or in abstract terms.
\item \emph{Perception} - Perception is the awareness of the character. It
  describes how well a character can describe his surroundings, either passively
  or actively. It determines how well he finds things that are hidden or well he
  notices small occurences around him.
\item \emph{Charisma} - Charisma determines how well a character can interact
  with other characters. This includes things as speech, facial expressions,
  gestures, articulation and expression.
\end{itemize}

Your character starts with \emph{five} (5) in all abilities. At the beginning
you may buy or sell ability ranks with CCP. Buying the next rank costs exactly
as many \emph{CCP} as the rank. So an increase from five to six would cost six
\emph{CCP}.

The reverse is also true if you lower an ability: You receive as much \emph{CCP}
by lowering as it would cost to raise the ability to the old value. So from
lowering an ability from five to four, you'd receive five \emph{CCP}.

At the beginning you may raise an ability not higher than seven, and may not
lower it beneath three. Later on in your adventuring career the GM may allow you
to increase an ability further. But that is up to the GM's discretion.

\subsection{Second: Pick a Background}

The game has your character learn skills to be useful. Backgrounds are a set of
predefined set of skills that fit to a certain theme, or job a character might
have learned throughout his career. They are optional, however, and experienced
players might wish to pick the skills for their character individually.

\end{document}
