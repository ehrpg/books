\chapter{Skills}

\subsection{Overview}

When a character peforms complex operations a skill is involved, that your
character has to learn first. Whether it be operating a vehicle, shooting a
firearm, climbing a steep cliff or trying to convince another character or
NPC. Each skill is associated with two abilities. If a skill check is made,
a trained character can add the modifier of those abilities to his roll.

Some skills are special, as in they require a lot of effort to learn and use
right, or require special materials, trainers or facilities to learn properly.
There are also speciality skills. These skills cannot be learned on their own,
but reflect a specialisation of an already existing skill. They help to
specialise and refine a character.

Each skill has a rank that goes from zero to ten. With zero meaning that the
character has just learned the skill and cannot draw upon any experience when
performing tasks. Ten represents absolute mastery.

\subsection{Learning new skills}

Normal skills take one \emph{CCP} to unlock, upon which the skill has a rank
of zero. If the skill is a special skill, the unlock costs are ten \emph{CCP}
instead.

Once the skill is learned, rank increases can be bought with the same exact
number of \emph{CCP} as the next rank.

\subsection{Using Skills}

Each task, against which your skill is pitched against, has a base difficulty.
If the base difficulty is at or below your skill level you may be able to
perform the task without rolling your dice in come circumstances. This means
that your character can perform this task given enough time and no adverse
external influences.

For example, an avid shooter can always hit his target on his range, if he's
rested and aims long enough. But this doesn't mean he can hit another target
on the same distance as easy in the heat of combat, fatigued by days of
fighting. As soon as external influences come into affect, you \emph{have to
  roll}.

If you have unlocked a skill you make a skill check rolling a ten-sided dice,
adding the appropriate ability modifiers, adding the skill rank and adding any
specialised skill ranks (if applicable).

If you have not unlocked the skill you simply roll a ten-sided dice and only
add negative ability modifiers to the roll.

When your result exceeds or is equal to the tasks difficulty (which is the base
difficulty, plus any situational hindrances) you succeed. Please see tasks and
\emph{Checks and Difficulties} for rules on causing minor, medium or major
successes or failures.

\subsection{Specialities}

Specialiaties are specialised sub-fields of another skill. They represent
niches, or specialised area your character can focus on. You are allowed to
take up to three specialised skills (if there are that many), depending on
the rank of the parent skill. Once you reach rank three on the parent skill
you make take one specialised skill, another at rank six and another at rank
nine.

Your specialised skill does not need unlocking, but you cannot have the rank
of the specialised skill above the rank of the base skill. If you perform a
skill check that involves that specialised skill you can add the rank of the
specialised skill to the skill check (alongside the skill rank of the base
skill).

For example, Hana a skilled physician wishes to specialise in Surgery. After
reaching Medicine at rank three, she buys two ranks of Surgery. Now when she
has to perform a surgery she can add the three from Medicine, and the two from
Surgery to the skill check for a total of plus five.

\subsection{Available Skills}

Below you will find a list of all the base skills available in game. It is of
course up the GM to adapt or add to the list below.

%% TODO: Sorting!

\emph{Melee Combat}: You are trained in the art of hand to hand combat. This
skill includes basic training with common melee weapons (knife, batons or
clubs). To hit an opponent you use \emph{Dexterity} and \emph{Strength}.
The damage depends on the weapon used.

\emph{Small Arms}: The small arms skill allows a character to aim, shoot, reload
and maintain certain fire arms. Many firearms that are can be learned without
special training, while some others (especially military grade hardware)
requires special training (see \emph{Light Weapons}). Firearms rely on
\emph{Perception} and \emph{Dexterity}. Rifles and handguns handle differently
and are thus split up in two separate speciality skills.

\emph{Small Arms (Handguns) (Speciality)}: Handguns include firearms such as
pistols, revolvers and some specific sub machine guns. They have limited range
and stoping power but are easily concealed.

\emph{Small Arms (Rifles) (Speciality)}: This category includes sporting and
hunting rifles, as well as shotguns.

\emph{Light Weapons}: This skill grants training in military grade hardware,
such as recoilless rifles, grenade launchers, mortars, light and heavy machine
guns. It uses \emph{Perception} and \emph{Dexterity}.

\emph{Heavy Weapons}: Heavy weapons are all weapons that cannot be effectively
carried into combat, and include things such as a howitzers, large mortars or
turrets.
