\chapter{Character Creation}
\label{chap:Character Creation}

\begin{wrapfigure}{r}{0.5\textwidth}
  \vspace{-20pt}
  \begin{mdframed}[skipabove=0pt,skipbelow=0pt,frametitlerule=true,frametitle={Character creation outline}]
    \begin{enumerate}
      \item \emph{Optional: Choose a background}
      \item Assign abilities
      \item Calculate secondary abilities
      \item Assign skills
    \end{enumerate}
  \end{mdframed}
  \vspace{-40pt}
\end{wrapfigure}

The following sections will guide you through character creation. Characters
in Endless Horizons are your average Joe and Jane that fight for survival
in the ultimate hostile space: the known universe.

\section{Character Creation Points - CCP}
\label{sec:2-Character Creation Points}

The entire process of creating and advancing your character is driven by so
called \emph{Character Creation Points} or \emph{CCP} for short. You receive
a given amount at the beginning to build your character, and receive more as
you progress in the game.

\emph{CCP} allow you buy ability increases, improve skills, and even allow you
get special features for your character called perks.

Most games will have you start with 50 \emph{CCP} at the beginning for character
creation. But it is always up the GM to fiddle with this number so that it fits
his campaign.

\section{Abilities}
\label{sec:2-Abilities}

Characters are defined by six major abilities. An ability can have a rank from
one to ten, with the average being five. Each ability also has a modifier. This
modifier represents if you gain a benefit or a disadvantage from your ability
rank. For each rank above five, you gain one modifier, and for each rank below
five you gain one minus modifier. So if your rank is three, you have a minus
two modifier, and if you rank is seven you have a plus two modifier.

The abilities are the following:

\begin{itemize}
\item \emph{Strength} - Strength represents your raw physical power. It
  determines not only how much a character can lift, drag or push. But also how
  far he can jump, or how much damage he does in melee combat.
\item \emph{Dexterity} - Determines your characters fine motor skills and how
  well your hand-eye coordination it is.
\item \emph{Constitution} - This ability determines how physically well your
  character is. It determines hit points, as well as endurance and overall
  fitness of your character.
\item \emph{Intelligence} - Intelligence represents how well a character learns,
  how well he can thing logically, or in abstract terms.
\item \emph{Perception} - Perception is the awareness of the character. It
  describes how well a character can describe his surroundings, either passively
  or actively. It determines how well he finds things that are hidden or
  well he notices small occurrences around him.
\item \emph{Charisma} - Charisma determines how well a character can interact
  with other characters. This includes things as speech, facial expressions,
  gestures, articulation and expression.
\end{itemize}

Your character starts with \emph{five} (5) in all abilities. A \emph{ten}
signalises the best there can be, for example ten strength would make you
strong man number one, or a \emph{ten} in constitution represents the peak of
human physique. Likewise values lower than five represent values lower than an
average person. A particular clumsy person might have a dexterity of
\emph{four} or even \emph{three}. Although ability scores below \emph{three}
represent some sort of physical disability. If you wish to lower your ability
below \emph{three} it is strongly advised to talk to your \emph{GM} before
doing so.

At the beginning you may buy or sell ability ranks with CCP. Buying the next
rank costs exactly as many \emph{CCP} as the rank. So an increase from five to
six would cost six \emph{CCP}.

The reverse is also true if you lower an ability: You receive as much \emph{CCP}
by lowering as it would cost to raise the ability to the old value. So from
lowering an ability from five to four, you'd receive five \emph{CCP}.

At the beginning you may raise an ability not higher than seven, and may not
lower it beneath three. Later on in your adventuring career the GM may allow you
to increase an ability further. But that is up to the GM's discretion.

\section{Secondary Abilities}
\label{sec:2-Secondary Abilities}

Apart from the primary abilities, \emph{Endless Horizons} has several secondary
abilities. These derive from the ones above and also change accordingly.

\subsection{Hit Points - HP}
\label{sub:2-Hit Points}

Hit Points - or \emph{HP} - are an abstract number that defines your characters
remaining health. If they reach zero your character dies. A characters hit
points are calculated by adding \emph{Constitution} and \emph{Strength} to
twenty-five (25). Note that the ranks are added, and not the modifiers.

\subsection{Speed}
\label{sub:2-Speed}

Speed is the amount of movement your character can do as part of an
\emph{action}. It is derived from adding the \emph{Dexterity} and
\emph{Strength} ability modifier together and add \emph{5}. A character with
zero speed points left is unable to move. The player may choose to expend one
of his actions to regain his speed points.

Please see the chapter about combat about how to use your Speed points.

\subsection{Carrying Capacity}
\label{sub:2-Carrying}

The maximum weight a character can carry for a longer time period is defined
by the character's \emph{Constitution} rank. Defining the maximum amount of kg
the character can carry is as easy as adding ten (10) to double your
\emph{Constitution} rank.

Do not confuse carrying with lifting, dragging or pushing defined by
the character's strength rank.

\section{Pick a Background}
\label{sec:2-Pick a Background}

The game has your character learn skills to be useful. Backgrounds are a set of
predefined set of skills that fit to a certain theme, or job a character might
have learned throughout his career. They are optional, however, and experienced
players might wish to pick the skills for their character individually.

Backgrounds should help you to get you started with creating your character, but
not limit you in the possibilities. If you feel that none of the background fits,
and you do not wish to pick the skills on your own, talk to the GM.

A background usually comes with an ability increase. This increase stacks with
whatever you picked in the first step. A background also comes with several
skills unlocked and at certain ranks.

You are free to select more than one background, but neither the skills nor the
ability increases stack. Only the highest one applies.

\section{Pick Skills}
\label{sec:2-Pick Skills}

Picking skills is the most important aspect of \emph{Endless Horizons}. There
are two types of skills: Special skills and normal skills. Special skills take
longer than normal skills to learn, or even require access to special equipment
or training facilities.

Each skill your character wishes to learn has to be unlocked. Unlocking costs
one, for normal skills, and ten \emph{CCP}s for special skills. Skill ranks
start at zero and go until ten. Each increment costs exactly as many
\emph{CCP}s the increment you wish to buy.

\section{Pick Perks}
\label{sec:3-Pick Perks}

Picking perks allows you to add a little bit of extra spice to a character.
They have prerequisites (ranks in skills, or a minimum rank in an ability,
or other perks) that your character must meet before you can take them.
Perks also cost \emph{CCP} of course.
