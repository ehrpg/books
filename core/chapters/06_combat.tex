\chapter{Combat}

This chapter deals with ranged and melee combat between characters. Ship to ship
combat is defined in its own section in the chapter that deals with Vessels.

\subsection{Turns}

Combat takes place in turns. Each player plots what he wishes to do, and
communicates this to the GM. A player may make two actions per turn. Valid
actions are:

\begin{itemize}
\item Move up to his speed.
\item Use a non-complicated item, like reloading a firearm, pull the pin from a
  grenade etc. See the description for the item to see how many actions it takes
  to operate it.
\item Speak up to ten words to communicate with fellow players.
\item Attack an enemy.
\item Prepare for an attack by the enemy.
\item Perform a skill check.
\end{itemize}

Upon all players have decided upon his actions and informed the GM, each player
rolls initiative to see when he gets to act. Rolling initiative is done by
rolling a ten sided die (d10) and adding the \emph{Dexterity} modifier and any
other relevant bonuses. Then all player characters do their actions in the
order they rolled initiative. Non player characters behave in the same way,
and also reroll their initiative each round.

In an alternate system (as preferred by Johannes) each participant of combat
rolls their initiative at the beginning, which then remains fixed throughout the
encounter. This system is preferable if you find that your encounters last too
long and you wish to speed them up.

\subsection{Surprise Attacks}

If the receiver of an attack (be it either ranged or melee) is unaware of the
attacker's presence the attacker makes what is called a \emph{surprise attack}.
The opponent is not allowed to dodge or block and has to take the attack.

\subsection{Melee Combat}

A player character can engage another in melee combat if they are close to each
other (up to a metre). To attack another opponent in melee roll a skill check
of the \emph{Melee Combat} skill. If it counts as a \emph{surprise attack} the
opponent is not allowed to dodge or block and must take the attack. The attacker
rolls for damage.

If the opponent is aware of the attack he may now use one of his actions this
rounds to dodge attack. If he has no more actions left, he cannot dodge. To
dodge attack the opponent makes a \emph{Dodge} check against the attacker's
attack check. If the defender exceeds the attack roll he has successfully evaded
the attack and takes no damage.

Should the opponent be out of actions for this turn, he may only block the
attack. The opponent rolls a ten sided die (d10) and adds \emph{Strength} and
\emph{Constitution} to the dice result. If he exceeds the attack roll, he has
successfully blocked the attack, and damage received is halved.
