\chapter{Combat}
\label{chap:Combat}

This chapter deals with ranged and melee combat between characters. Ship to ship
combat is defined in its own section in the chapter that deals with Vessels.

\section{Turns}

Combat takes place in turns. Each player plots what he wishes to do, and
communicates this to the GM. A player may make two actions per turn. Valid
actions are:

\begin{itemize}
\item Move up to his speed.
\item Use a non-complicated item, like reloading a firearm, pull the pin from a
  grenade etc. See the description for the item to see how many actions it takes
  to operate it.
\item Speak up to ten words to communicate with fellow players.
\item Attack an enemy.
\item Prepare for an attack by the enemy.
\item Perform a skill check.
\end{itemize}

Upon all players have decided upon his actions and informed the GM, each player
rolls initiative to see when he gets to act. Rolling initiative is done by
rolling a ten sided die (d10) and adding the \emph{Dexterity} modifier and any
other relevant bonuses. Then all player characters do their actions in the
order they rolled initiative. Non player characters behave in the same way,
and also reroll their initiative each round.

In an alternate system (as preferred by Johannes) each participant of combat
rolls their initiative at the beginning, which then remains fixed throughout the
encounter. This system is preferable if you find that your encounters last too
long and you wish to speed them up.

\subsection{Surprise Attacks}

If the receiver of an attack (be it either ranged or melee) is unaware of the
attacker's presence the attacker makes what is called a \emph{surprise attack}.
The opponent is not allowed to dodge or block and has to take the attack.

\section{Melee Combat}

A player character can engage another in melee combat if they are close to each
other (up to a metre). To attack another opponent in melee roll a skill check
of the \emph{Melee Combat} skill. If it counts as a \emph{surprise attack} the
opponent is not allowed to dodge or block and must take the attack. If attacker
hits, he rolls for damage. See the equipment section for weapons and their
damage.

If the opponent is aware of the attack he may now use one of his actions this
round to dodge the attack. If he has no more actions left, he cannot dodge. To
dodge the attack the opponent makes a \emph{Dodge} check against the attacker's
attack check. If the defender exceeds the attack roll he has successfully evaded
the attack and takes no damage.

Should the opponent be out of actions for this turn, he may only block the
attack. The opponent rolls a ten sided die (d10) and adds \emph{Strength} and
\emph{Constitution} modifiers to the dice result. If he exceeds the attack roll,
he has successfully blocked the attack, and damage received is halved.

\section{Ranged Combat}

Ranged combat works differently. A player chooses to engage another in ranged
combat, takes aim and fires. The player who is being targeted has no chance
usually has little change to evade (except on extreme distances). The difficulty
of landing a successful hit depends on several factors:

\begin{itemize}
\item \emph{Size}: The difficulty increases if the target is smaller or
  decreases if the target is bigger.
\item \emph{Range}: Depending on the weapon in question, targets get harder to
  hit at longer ranges.
\item \emph{Stances}: Various stances can decrease the difficulty. A person
  firing fully automatic while standing has a lower chance of hitting than a
  person who fires single shots from the prone position.
\item \emph{Movement}: A moving target is hard to hit, especially at further
  ranges.
\item \emph{Environmental}: Bad sight, strong winds can all affect on how well
  a character can hit a target.
\end{itemize}

The game master calculates the difficulty of the task depending on these factors
and player rolls an appropriate skill check. If he reaches it or exceeds the
difficulty he scores a hit.

\subsection{Cover}

The best way to protect oneself from fire is to hide beneath or behind cover.
Depending on the material of the cover bullets may still be able to penetrate
the cover and cause damage to the people behind the cover.

For example most bullets can penetrate a wooden cover, whereas a .50 BMG is even
capable of tearing through brick walls.

\subsection{Size}

The game separates targets into size categories. An average human standing is
considered a \emph{medium} target. The same average human that is either sitting
on the floor or kneeling is considered a \emph{small} target. The same person
in the prone position is considered \emph{tiny}. These size categories do not
change depending on the distance from the shooter. They just help to categorize
the targets. For example a big dog would be considered a \emph{small} target,
while a sneaking would count as \emph{tiny}.

Targets larger than a human also have their own size categories. A car, for
example is considered a \emph{large} target. So are some other vehicles that
fit roughly the same size. Bigger targets are considered \emph{huge}, for
example large vehicles (main battle tanks, MRAPs, IVFs) or even space ships.

\subsection{Range}

The range to a target is always given rough estimate in metres. The game master
should avoid making a science of it, and should operate in rough estimates.

The gun play mechanics then categorize these ranges into what is called an
\emph{effective range}\footnote{Many readers, who know their way around
  firearms: Sorry for inducing cringe.} This is a metric used in conjunction
with firearms to determine how effective they are at certain ranges. The authors
know that firearms are more complicated than that, and much of it depends on the
skill of the shooter involved. Still this metric is used for abstraction and for
balancing reasons.

The \emph{effective range} is broadly defined as the distance an average shooter
should be able to hit a \emph{medium} target half of the time. Other ranges are
derived from this in fractions:

\begin{itemize}
\item \emph{Maximum}: Between \emph{effective range} and half the
  \emph{effective range}.
\item \emph{Far}: Between \emph{maximum} and a quarter of the
  \emph{effective range}.
\item \emph{Medium}: Between \emph{far} and a eigth of the
  \emph{effective range}.
\item \emph{Close}: Everything below an eights of the \emph{effective range}.
\end{itemize}

For example, let's assume a Glock 17 has an effective range of 50
metres. The ranges are thus as following: \emph{Maximum} is between 50
and 25 metres. \emph{Far} between 25 and 12.5 metres. \emph{Medium}
between 12.5 metres and six metres, and \emph{close} is everything
closer than six metres.

The size category of the target and the range category are combined and lead
to the following table of penalties for the difficulty task:

\begin{center}
  \begin{tabular}{| l | l | l | l | l |}
    \hline
    \,        & Close & Medium & Far & Maximum \\ \hline
    Tiny      & +3    & +6     & +12 & +18    \\ \hline
    Small     & +2    & +4     & +8  & +12    \\ \hline
    Medium    & +1    & +2     & +4  & +6     \\ \hline
    Large     & -12   & -8     & -4  & -2     \\ \hline
    Huge      & -18   & -12    & -6  & -3     \\
    \hline
  \end{tabular}
\end{center}

\subsubsection{Movement}

Movement is also categorized into four categories: \emph{still}, \emph{subtle},
\emph{slow} and \emph{fast}. A normal human being standing is considered
moving \emph{subtly} as no person normally stands perfectly still. Objects
may as well stand perfectly \emph{still}, as in their movement is not
noticable. So do living beings who are perhaps bound, unconscious or otherwise
incapable of moving. \emph{Slow} movement incorporates anything at the speed
of an average human walking. \emph{Fast} is anything that moves at a speed of
an average human sprinting or above.

The penalties are as followed:

\begin{center}
  \begin{tabular}{| l | l | l | l | l |}
    \hline
    \,      & Still & Subtle & Slow & Fast \\ \hline
    Penalty & -2    & 0      & +2   & +4   \\
    \hline
  \end{tabular}
\end{center}

These penalties apply for both a moving target, and a shooter that is moving. So
a player who is running and shooting at a target that is also running as a
cumulative penalty of plus eight (+8).

\subsection{Stance}

A person firing can be in one of the following firing stances: \emph{standing},
\emph{kneeling}, \emph{prone}, and \emph{other}. Standing also makes the person
in question a medium target, \emph{kneeling} makes him a small and \emph{prone}
a tiny target.

The stances above assume that the character is in the proper firing positions.
A person who just fell down to the floor on his back is \emph{not} considered to
be in prone firing position. Although the person would still be considered a
tiny target.

The penalties for the stances are as followed:

\begin{center}
  \begin{tabular}{| l | l |}
    \hline
    Stance   & Penalty \\ \hline
    Standing & -1      \\ \hline
    Kneeling & -2      \\ \hline
    Prone    & -3      \\
    \hline
  \end{tabular}
\end{center}

\subsection{Taking Aim}

A character may expend an action to take careful aim. The difficulty is reduced
by two (2) for each action he chooses to take aim for, up to a maximum of 6 for
three actions.

\subsection{Bursts and Full Auto}

Each weapon which is capable of burst firing or fully automatic has a recoil
modifier. Each subsequent bullet adds this recoil modifier to the difficulty.
Regardless only one attack roll is made for the burst. As many bullets of the
burst hit as the attack roll meets or exceeds in difficulty. While the
progression is linear (per bullet) in a real life scenario it is perfectly
reasonable for the first bullet to hit, the second to miss and the third to
hit again.

Example: A player wishes to shoot a three round burst at a target. The
difficulty is determined to be 16 and the recoil modifier of the gun is 2.
The player rolls a 19. This means that the first bullet (difficulty 16) hits,
the second does as well (as the player's 19 is above 16 plus 2) but the third
bullet does not (19 is below 20).

\subsection{Suppressive Fire}

Sometimes it is necessary to keep the enemy sitting ducks. Usually to allow
someone else to flank around, or to cover the get away of others. With
suppressive fire you have to fire a certain amount bullets towards the
enemies you wish to suppress for each action you wish to suppress them. This
includes your actions, those of your allies and the actions of the enemy. The
amount of bullets required is determined by the \emph{GM}. Depending on the
situation. The \emph{GM} may even allow you to fire and aim at an enemy out of
turn if they leave their cover while you suppress them.

If you run out of bullets and have to reload you have to await your turn to
reload and the suppression ends.

The \emph{GM} has to determine the bullets required. For example, covering a
simple narrow corridor (with enemies left and right of the door) usually
requires just two bullets per action. While a wide open area with natural
cover and enemies advancing may require a short burst per cover.

Despite common misconception it is not required to fire full auto to achieve
a suppression effect. If a sharp shooter covers a narrow side alley and takes
single shots at anyone who wishes to cross, the enemies will still think twice
before crossing.

\subsection{Reloading}

Reloading a firearms takes a certain amount of actions. If not otherwise noted
in the description of the firearm, a reload time of one action is assumed.

\subsection{Malfunctions}

Firearms are mechanical devices that can fail. Wear and tear, or inproper
cleaning and handling can contribute to these malfunctions or stoppages. There
are variety of possible malfunctions, ranging from temporary to relative save
malfunctions to potentially dangerous ones that can permanently damage the gun
or cause injury or death.

A minor fault is a \emph{failure to feed} (FTF) when the firearm fails to feed
the next cartridge into the chamber from a magazine. It can also be caused by
not holding the fire arm properly (``limp wristing''). It can usually be cleared
in one action and does not immediately threaten the shooter.

A stovepipe (\emph{failure to eject}) occours when the spent casing is not
properly ejected from the firearm, causing it to jam the ejection port. It
presents no immediate danger to the shooter and can be cleared in one action.

A severe major fault is a so called \emph{squib load} where the projectile does
not have enough force to exit the barrel and becomes stuck. A subsequent bullet
fired afterwards can cause catastrophic failure, causing damage to the firearm
and injury to the shooter.

\subsection{Determining Difficulty}

The base difficulty for ranged attacks is five (5). Then you combine the size
category and the range category and add it. Then you add a movement penalty,
if the target is moving or the shooter is moving. Then you add stance modifiers,
any environmental modifiers and any miscellaneous modifiers.

\emph{Example}: A player wishes to fire a three round burst with his rifle
(recoil modifier of 2) (range 300 metres) at a small target roughly 100 metres
out. It is dusk and the game master decides that he has an additional penalty
of +2 for from low light conditions. The player as a dexterity and perception
modifier of +1, and three ranks in the appropriate skill. The difficulty is
determined as following:

\[
DC = 5 \ub{+ 8}{range} \ob{+ 1}{low light} \ub{- 1}{stance} = 13
\]

The player rolls the following:

\[
1d10 = 8 \ub{+ 1}{dexterity} \ob{+ 1}{perception} \ub{+ 3}{skill} = 13
\]

And scores one hit. The other two of his bullets miss.

\section{Thrown Weapons}

Thrown weapons (throwing knives, axes and hand grenades) are handled just like
other ranged weapons, except that a different skill is used in the checks:
\emph{Throwing}. Thrown weapons also have an \emph{effective range} and cause
damage just like a bullet from a firearm would.

\subsection{Grenades}

Throwing a grenade at a specific area (for example next or behind a cover) is
usually considered a range task with a \emph{medium} target (the area). Any
special circumstances might reduce or widen the size. For example throwing a
grenade into a window would cause the target to be \emph{small} (for a small
window), or casting a grenade into a small man-hole might even be considered
a \emph{tiny} target.

Most hand held explosive grenades have what is called an \emph{effective} and
an \emph{casualty} radius. Everyone within the \emph{effective} radius, that is
not protected by grenade proof cover, dies. Within the \emph{casualty} radius
a grenade deals enough damage to kill or incapacitate all of the targets.

Grenades cause two types of damage: \emph{bludgeoning} from the explosive shock
wave, and piercing \emph{piercing} from the flying shrapnel. Due to the massive
amounts of shrapnel only full cover or full body armour can protect against the
piercing damage.

\section{Damage}

The weapon used always determines the amount and type of damage that can be
caused to another. A knife for example can deal piercing or slashing damage
depending how it's being used. A firearm usually causes piercing and
bludgeoning damage. Hits with bare fists may just cause bludgeoning damage.

Other forms of damage are of course also possible, for example a player can
take fire damage from an open flame, or acid damage from being exposed to
acids or other corrosive fluids.

\subsection{Zones and Wounds}

When a player hits another, either with a ranged or melee weapon, the attacking
player rolls two six sided die (2d6) to determine where he scores a hit.

The target then determines if he has clothing or armor that protects him in this
region and substracts any relevant damage reductions from the incoming damage.
If any damage is left the target resolves any effects of the target area (for
example damage multiplier) and substracts that number from his hit points. If
the target reaches zero or below zero hit points total, the target is dead.

If the target survives it, he then has to roll another check to see if the
injury causes a major wound. The difficulty is determined as followed:

\[
DC = \ub{5}{base} + \ob{dmg}{Damage taken}
\]

The \emph{Damage taken} is the actual damage you took, after any reductions
through armour, and before any additional damage multipliers.

Against the following check:

\[
1d10 + \ub{con}{Constition score}
\]

\begin{table}
  \caption{Wound zones for a human, or human-like creature}
  \begin{center}
    \begin{tabular}{|l|l|l|l|}
      \hline
      Result& Zone         & Effect                           & Major Wound     \\ \hline
      2     & Vital Organ  & \emph{Special}, \(Damage * 6\)   &                 \\ \hline
      3,4   & Legs         &                                  & -2 Speed        \\ \hline
      5,6,7 & Torso        & \(Damage * 2\)                   & -2 Constituion  \\ \hline
      8,9   & Arms         &                                  & -2 Dexterity    \\ \hline
      10,11 & Head         & \(Damage * 3\)                   & -2 Perception   \\ \hline
      12    & Vital Organ  & \emph{Special}, \(Damage * 6\)   &                 \\
      \hline
    \end{tabular}
  \end{center}
\end{table}

\subsection{Vital Organs}

Most creatures, animals (including humans) have vital organs. Any damage
to those areas is usually fatal. For humans these areas would be the brain
and the heart. Damage to the vital organs follows special rules, usually
depending on the creature involved. However if - after applying these special
rules - any damage is done to the vital organ this damage is multiplied by
\emph{six} (6).

\subsubsection{Humans}

Humans have two vital organs: The \emph{heart} and the \emph{brain}. Both
the brain and the heart are usually protected either by the skull or the
rib cage. Any bludgeoning damage below \emph{ten} (10) is automatically
converted into a hit towards the surrounding hit zone (torso for
heart, and head for brain). Slashing damage cannot harm either of those
vital two organs unless they are exposed. Any piercing damage above \emph{two}
(2) can easily penetrate the rib cage or skull and does full damage. Any
piercing damage below \emph{two} is stopped by the bone and negated.
