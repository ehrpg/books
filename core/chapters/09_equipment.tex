\chapter{Equipment}

\section{Currency}

The world of Endless Horizon knows only one currency: Credit (C) for short.
Everything is payed for and with \emph{Credit}.

\section{Firearms}

This section lists all available firearms. Each firearm uses a magazine to feed
(be it internally or externally detachable) and fires some sort of cartridge.
Different firearms use different calibers, and for each caliber different
ammunition is available. Most of the time these are normal, hollow point (for
increased effectiveness non-armoured targets) and armor piercing (for more
damage against armour.

If a firearm is not listed in the section and your players wishes to use it,
just add them. This list is not exhaustive for the massive amounts of available
firearms in the world today.

\subsection{Magazines}

Many modern firearms (especially handguns, rifles) come with detachable bog
magazines. These are most commonly not interchangable between firearms unless
explicitely noted. Many firearms also come with internal magazines, for example
many pump action shotguns come with an internal magazine. Another good example
are revolvers.

Magazines are cheap to come by, and usually cost a tenth of the firearm they
belong to.

It is up to the player to manage, reload and maintain his magazines. It is not
possible to reload a magazine during combat but if spare bullets are available
magazines can be reloaded during downtime (travel with a shuttle etc.).

\subsection{Firing modes}

Most fire arms are single action (meaning one trigger pull fires one bullet).
But many also come with burst (one trigger pull fires multiple bullets) or
even fully automatic. The modes that are available depends on the firearm or
even sometimes on the model of a specific firearm.

The table lists modes with abbreviations. \textbf{S} stands for single shot,
\textbf{B} stands for burst (three round burst unless otherwise noted),
\textbf{A} stands for automatic, and \textbf{R} stands for repeating, meaning
that a mechanism has to be operated to cycle the next round into the chamber.

\subsection{Cartridges}

Each firearm fires a certain caliber. The bullet defines the damage that is
done, while the firearm describes additional characteristics such as effective
range, firemodes and more. Many cartridges come in different variations, such as
normal, sub sonic, tracer; and may come with different bullets (hollow point,
armor penetrating steel tip). Down below is a list of cartridges that are
currently available in \emph{Endless Horizons}:

\begin{center}
  \begin{tabular}{| l | l | l | l | l | l |}
    \hline
    \textbf{Cartridge}  & \textbf{Type}   & \textbf{Piercing} &
    \textbf{Bludgeoning} & \textbf{Cost}  & \textbf{Notes}        \\ \hline

    9x18    & Normal & 1d4  & 1d4  & 10 (50 Rnds.) & Stopped by Class II \\ \hline
    \,      & AP     & 1d4  & 1d4  & 15 (50 Rnds.) & Stopped by Class III \\ \hline
    \,      & HP     & 1d6  & 1d6  & 13 (50 Rnds.) & Stopped by Class I \\ \hline
    9x19    & Normal & 1d6  & 1d6  & 15 (50 Rnds.) & Stopped by Class II \\ \hline
    \,      & AP     & 1d6  & 1d8  & 20 (50 Rnds.) & Stopped by Class III \\ \hline
    \,      & HP     & 1d8  & 1d8  & 20 (50 Rnds.) & Stopped by Class II \\ \hline
    .40 ACP & Normal & 1d8  & 1d8  & 23 (50 Rnds.) & Stopped by Class II, Subsonic \\ \hline
    \,      & AP     & 1d8  & 1d8  & 25 (50 Rnds.) & Stopped by Class III \\ \hline
    \,      & HP     & 2d4  & 2d4  & 25 (50 Rnds.) & Stopped by Class II, Subsonic \\ \hline
    4.6x30  & Normal & 2d4  & 2d4  & 40 (50 Rnds.) & Stopped by Class III \\ \hline
    \,      & AP     & 2d4  & 2d4  & 60 (50 Rnds.) & Stopped by Class IV \\ \hline
    \,      & HP     & 2d6  & 2d6  & 50 (50 Rnds.) & Stopped by Class II \\ \hline
    5.7x28  & Normal & 2d4  & 2d4  & 40 (50 Rnds.) & Stopped by Class III \\ \hline
    \,      & AP     & 2d4  & 2d4  & 60 (50 Rnds.) & Stopped by Class IV \\ \hline
    \,      & HP     & 2d6  & 2d6  & 50 (50 Rnds.) & Stopped by Class II \\ \hline
    9x21    & Normal & 2d4  & 2d4  & 40 (50 Rnds.) & Stopped by Class III \\ \hline
    \,      & AP     & 2d4  & 2d4  & 60 (50 Rnds.) & Stopped by Class IV \\ \hline
    \,      & HP     & 2d6  & 2d6  & 50 (50 Rnds.) & Stopped by Class II \\ \hline

  \end{tabular}
\end{center}

\begin{center}
  \begin{tabular}{| l | l | l | l | l | l |}
    \hline
    \textbf{Cartridge}  & \textbf{Type}   & \textbf{Piercing} &
    \textbf{Bludgeoning} & \textbf{Cost}  & \textbf{Notes}        \\ \hline

    7.62x39  & Normal  & 2d10  & 2d10  & 70 (30 Rnds.)  & Stopped by Class IV \\ \hline
    \,       & AP      & 2d10  & 2d20  & 120 (30 Rnds.) & Stopped by Class V \\ \hline
    5.56x45  & Normal  & 2d8   & 2d8   & 80 (30 Rnds.)  & Stopped by Class III \\ \hline
    \,       & AP      & 2d8   & 2d8   & 100 (30 Rnds.) & Stopped by Class IV \\ \hline
    \,       & HP      & 2d10  & 2d10  & 120 (30 Rnds.) & Stopped by Class III \\ \hline
    5.45x39  & Normal  & 2d8   & 2d8   & 70 (30 Rnds.)  & Stopped by Class III \\ \hline
    \,       & AP      & 2d8   & 2d8   & 90 (30 Rnds.)  & Stopped by Class IV \\ \hline
    \,       & HP      & 2d10  & 2d10  & 110 (30 Rnds.) & Stopped by Class III \\ \hline
    7.62x51  & Normal  & 2d10  & 2d10  & 100 (20 Rnds.) & Stopped by Class IV \\ \hline
    \,       & AP      & 2d10  & 2d10  & 130 (20 Rnds.) & Stopped by Class V \\ \hline
    \,       & HP      & 3d6   & 3d6   & 150 (20 Rnds.) & Stopped by Class IV \\ \hline
    7.62x54  & Normal  & 2d10  & 2d10  & 100 (20 Rnds.) & Stopped by Class IV \\ \hline
    \,       & AP      & 2d10  & 2d10  & 130 (20 Rnds.) & Stopped by Class V \\ \hline
    \,       & HP      & 3d6   & 3d6   & 150 (20 Rnds.) & Stopped by Class IV \\ \hline
    .50 BMG  & Normal  & 3d8   & 3d8   & 200 (10 Rnds.) & Stopped by Class VI \\ \hline
    \,       & AP      & 3d8   & 3d8   & 250 (10 Rnds.) & Stopped by Class VII \\ \hline
    \,       & HP      & 3d10  & 3d10  & 250 (10 Rnds.) & Stopped by Class VI \\ \hline
    \,       & HEI/AP  & 4d8   & 4d8   & 350 (10 Rnds.) & Stopped by Class VII \\ \hline
    12.7x108 & Normal  & 3d8   & 3d8   & 200 (10 Rnds.) & Stopped by Class VI \\ \hline
    \,       & AP      & 3d8   & 3d8   & 250 (10 Rnds.) & Stopped by Class VII \\ \hline
    \,       & HP      & 3d10  & 3d10  & 250 (10 Rnds.) & Stopped by Class VI \\ \hline
    \,       & HEI/AP  & 4d8   & 4d8   & 350 (10 Rnds.) & Stopped by Class VII \\ \hline

  \end{tabular}
\end{center}

\begin{center}
  \begin{tabular}{| l | l | l | l | l | l |}
    \hline
    \textbf{Cartridge}  & \textbf{Type}   & \textbf{Piercing} &
    \textbf{Bludgeoning} & \textbf{Cost}  & \textbf{Notes}        \\ \hline

    12G     & Buckshot & 4d4  & 1d4 & 20 (12 Shells) & Stopped by Class II \\ \hline
    \,      & Slug     & 3d4  & 2d4 & 25 (12 Shells) & Stopped by Class III \\ \hline
    \,      & Dart     & 3d4  & 3d4 & 30 (12 Shells) & Stopped by Class III \\ \hline

  \end{tabular}
\end{center}

\subsection{Pistols}

Pistols (or handguns) are compact weapons usually used in close quarter combat.
While most militaries use them as secondary weapons (alongside more powerful
rifles) many civilians or contractors rely on them mainly for their security
and defense.

Most pistols are semi automatic, feeding from a detachable box magazine.

\begin{table}
  \caption{9x18 Handguns}
  \begin{center}
    \begin{tabular}{| l | l | l | l | l | l | l | l |}
      \hline
      \textbf{Name} & \textbf{Cartridge} & \textbf{Range} &
      \textbf{FM} & \textbf{Magazine} & \textbf{Recoil} &
      \textbf{Cost} & \textbf{Notes} \\ \hline

      Makarov PM & 9x18mm & 50m & S & 8  & 1 & 90  & \\ \hline
      Fort-12    & 9x18mm & 50m & S & 12 & 2 & 160 & \\ \hline
      MP-448     & 9x18mm & 50m & S & 12 & 2 & 160 & \\ \hline

    \end{tabular}
  \end{center}
\end{table}

\begin{table}
  \caption{9x19 Handguns}
  \begin{center}
    \begin{tabular}{| l | l | l | l | l | l | l | l |}
      \hline
      \textbf{Name} & \textbf{Cartridge} & \textbf{Range} &
      \textbf{FM} & \textbf{Magazine} & \textbf{Recoil} &
      \textbf{Cost} & \textbf{Notes} \\ \hline

      Walther P99  & 9x19mm & 60m & S  & 17 & 2 & 250 & \\ \hline
      Steyr M9 A1  & 9x19mm & 50m & S  & 17 & 2 & 230 & \\ \hline
      H\&K VP9      & 9x19mm & 50m & S  & 15 & 2 & 200 & \\ \hline
      Glock 17     & 9x19mm & 50m & S  & 17 & 2 & 230 & \\ \hline
      Glock 18     & 9x19mm & 50m & SA & 33 & 3 & 500 & Select-fire \\ \hline
      SIG P226     & 9x19mm & 50m & S  & 15 & 2 & 210 & \\ \hline
      Beretta 92FS & 9x19mm & 50m & S  & 18 & 2 & 260 & \\ \hline
      MP-443       & 9x19mm & 50m & S  & 18 & 2 & 260 & \\ \hline

    \end{tabular}
  \end{center}
\end{table}

\begin{table}
  \caption{.45ACP Handguns}
  \begin{center}
    \begin{tabular}{| l | l | l | l | l | l | l | l |}
      \hline
      \textbf{Name} & \textbf{Cartridge} & \textbf{Range} &
      \textbf{FM} & \textbf{Magazine} & \textbf{Recoil} &
      \textbf{Cost} & \textbf{Notes} \\ \hline

      M1911        & .45 ACP & 50m & S &  7 & 3  & 150 & \\ \hline
      H\&K USP      & .45 ACP & 60m & S & 12 & 3  & 350 & \\ \hline
      Hi-Point JHP & .45 ACP & 50m & S & 10 & 3  & 250 & \\ \hline
      Glock 21     & .45 ACP & 50m & S & 13 & 3  & 250 & \\ \hline

    \end{tabular}
  \end{center}
\end{table}

\subsection{Shotguns}

Most shotguns are of the pump-action variety and come with an internal tube that
holds a fixed amount of ammunition. The TOS-34 is a stand in for most break-top
shotguns (regardless of many barrels they have). There are simply too many to
list them all.

\begin{table}
  \caption{Shotguns}
  \begin{center}
    \begin{tabular}{| l | l | l | l | l | l | l | l |}
      \hline
      \textbf{Name} & \textbf{Cartridge} & \textbf{Range} &
      \textbf{FM} & \textbf{Magazine} & \textbf{Recoil} &
      \textbf{Cost} & \textbf{Notes} \\ \hline

      Remington M870 & 12G & 90m & S  & 7+1 & 4 & 310 & Pump-action, tube \\ \hline
      Mossberg 500   & 12G & 90m & S  & 8+1 & 4 & 350 & Pump-action, tube \\ \hline
      TOS-34         & 12G & 100 & S  & 2   & 4 & 110 & Double-barrel \\ \hline
      SPAS-12        & 12G & 80m & SA & 8+1 & 5 & 400 & Internal tube \\ \hline
      AA-12          & 12G & 70m & SA & 10  & 3 & 600 & \\ \hline
      Saiga-12       & 12G & 70m & SA & 10  & 4 & 550 & \\ \hline

    \end{tabular}
  \end{center}
\end{table}

\subsection{Submachine Guns}

Submachine guns fire a pistol caliber and very concealable and deadly in close
quarters. They are often issued to support personel, vehicle operators and
security detail that need a compact but deadly and efficient weapon.

\begin{table}
  \caption{Submachine Guns}
  \begin{center}
    \begin{tabular}{| l | l | l | l | l | l | l | l |}
      \hline
      \textbf{Name} & \textbf{Cartridge} & \textbf{Range} &
      \textbf{FM} & \textbf{Magazine} & \textbf{Recoil} &
      \textbf{Cost} & \textbf{Notes} \\ \hline

      MP5            & 9x19   & 150m & SBA & 30 & 3 & 500 & \\ \hline
      UMP-45         & .45ACP &  70m & SBA & 25 & 5 & 600 & \\ \hline
      MP7A1          & 4.6x30 & 200m & SA  & 30 & 3 & 750 & \\ \hline
      FN P90         & 5.7x28 & 200m & SA  & 50 & 3 & 800 & \\ \hline
      SR-2M          & 9x21   & 200m & SA  & 30 & 3 & 750 & \\ \hline
      Scorpion EVO 3 & 9x19   & 100m & SA  & 30 & 3 & 600 & \\ \hline
      KRISS Vector   & .45ACP &  60m & SA  & 25 & 3 & 700 & \\ \hline
      SIG MPX        & 9x19   &  90m & SA  & 30 & 3 & 550 & \\ \hline

    \end{tabular}
  \end{center}
\end{table}

\subsection{Rifles}

Rifles range from combat rifles, assault rifles to bolt action rifles. Almost
all armies use one as their main infantry weapon.

The AN-94 comes with a special burst, where the effective recoil of the weapon
is 1 for the second bullet in burst.

The AS Val comes with an integrated silencer. If removed, the gun will not
cycle anymore and loose half of its effective range.

\begin{table}
  \caption{Assault Rifles}
  \begin{center}
    \begin{tabular}{| l | l | l | l | l | l | l | l |}
      \hline
      \textbf{Name} & \textbf{Cartridge} & \textbf{Range} &
      \textbf{FM} & \textbf{Magazine} & \textbf{Recoil} &
      \textbf{Cost} & \textbf{Notes} \\ \hline

      SA Vz. 58    & 7.62x39 & 500m & SA  & 30 & 6 & 500 & \\ \hline
      AKM          & 7.62x39 & 350m & SA  & 30 & 7 & 300 & \\ \hline

      M16          & 5.56x45 & 550m & SB  & 30 & 5 & 450 & \\ \hline
      Steyr AUG A3 & 5.56x45 & 500m & SA  & 30 & 6 & 300 & \\ \hline
      CZ-805 BREN  & 5.56x45 & 500m & SA  & 30 & 5 & 400 & \\ \hline
      H\&K G36     & 5.56x45 & 600m & SA  & 30 & 5 & 450 & \\ \hline
      M4A1         & 5.56x45 & 450m & SA  & 30 & 5 & 500 & \\ \hline
      H\&K 416     & 5.56x45 & 500m & SA  & 30 & 5 & 550 & \\ \hline

      AK-12        & 5.45x39 & 600m & SB(2)A & 30 & 5 & 660 & \\ \hline
      AK-74M       & 5.45x39 & 500m & SA     & 30 & 5 & 550 & \\ \hline
      AKS-74U      & 5.45x39 & 350m & SA     & 30 & 5 & 450 & \\ \hline
      AN-94        & 5.45x38 & 600m & SB(2)A & 30 & 5 (1) & 900 & \\ \hline

      OTs-14       & 9x39mm  & 300m & SA     & 20 & 4 & 800 & \\ \hline
      AS Val       & 9x39mm  & 300m & SA     & 20 & 4 & 900 & \\ \hline
      SR-3         & 9x39mm  & 200m & SA     & 20 & 4 & 700 & \\ \hline

    \end{tabular}
  \end{center}
\end{table}


\begin{table}
  \caption{High-Caliber Rifles}
  \begin{center}
    \begin{tabular}{| l | l | l | l | l | l | l | l |}
      \hline
      \textbf{Name} & \textbf{Cartridge} & \textbf{Range} &
      \textbf{FM} & \textbf{Magazine} & \textbf{Recoil} &
      \textbf{Cost} & \textbf{Notes} \\ \hline


      H\&K 417      & 7.62x51  &  700m & SA & 20 & 7  &  700 & \\ \hline
      Mk 14 EBR     & 7.62x51  &  800m & SA & 20 & 7  &  750 & \\ \hline
      Galil ACE 53  & 7.62x51  &  600m & SA & 20 & 7  &  650 & \\ \hline
      FN SCAR-H     & 7.62x51  &  800m & SA & 20 & 7  &  700 & \\ \hline
      M40A5         & 7.62x51  &  800m & SA & 10 & 8  &  600 & \\ \hline

      SVD           & 7.62x54  &  800m & S  & 20 & 8  &  500 & \\ \hline
      SVU           & 7.62x54  & 1000m & S  & 20 & 8  &  700 & \\ \hline

      Barret M82    & .50 BMG  & 1800m & S  & 10 & 12 & 1100 & \\ \hline
      Steyr HS .50  & .50 BMG  & 1800m & S  &  1 & 11 &  900 & \\ \hline

      OSV-96        & 12.7x108 & 2000m & S  &  5 & 12 & 1300 & \\ \hline
      KSVK          & 12.7x108 & 1500m & S  &  5 & 12 & 1200 & \\ \hline
    \end{tabular}
  \end{center}
\end{table}


\subsection{Machine Guns}

Light machine guns offer a big magazine capacity (often fed from a belt) and
capability for long sustained fire. While carrying and firing from the shoulder
is possible, they are far more effective when fired from a bipod or when mounted
on a svivel.

Most belt fed machien guns feed from a chain, which either disintegrates after
being fired or not. If it doesn't disintegrate it can be refilled.

\begin{table}
  \caption{High-Caliber Rifles}
  \begin{center}
    \begin{tabular}{| l | l | l | l | l | l | l | l |}
      \hline
      \textbf{Name} & \textbf{Cartridge} & \textbf{Range} &
      \textbf{FM} & \textbf{Magazine} & \textbf{Recoil} &
      \textbf{Cost} & \textbf{Notes} \\ \hline


      H\&K MG4     & 5.56x45 &  500m & SA & 100 & 5 & 1200 & \\ \hline
      FN 240B      & 7.61x51 &  800m & SA & 100 & 9 & 1500 & \\ \hline
      PKP Pecheneg & 7.62x54 & 1200m & SA & 100 & 9 & 2100 & \\ \hline

    \end{tabular}
  \end{center}
\end{table}

\section{Body Armor}

Body armor comes in many different shapes and sizes. The most common are metal
plates that are inserted into what is called a ``chest carrier''. These plates
then stop the bullets from entering and absorb the kinetic force of the bullet.
The armor usually covers the most vital part of body (i.e. the chest and back)
and come with a specific rating. If a specificied caliber is stopped by the
armor the target takes no damage instead. The damage is absorbed by the armor
reaches zero hit points it stops being effective.

Armor is classified into various \emph{tiers} determing what kinds of threats
it stops. Please see the calibers to see what kind of armor tier a specific
caliber penetrates. All armor is capable of defending a player from slashing
damage from knives and other small blades or thrown weapons.

All plates require a carrier.

\begin{table}
  \caption{Armor}
  \begin{center}
    \begin{tabular}{| l | l | l | l | l |}

      \hline
      \textbf{Armor} & \textbf{Class} & \textbf{HP} &
      \textbf{Cost} & \textbf{Notes} \\ \hline

      Kevlar Vest     &  II & 40 & 150 & \\ \hline
      Plate Class I   &   I & 30 & 100 & \\ \hline
      Plate Class II  &  II & 40 & 150 & \\ \hline
      Plate Class III & III & 50 & 250 & \\ \hline
      Plate Class IV  &  IV & 60 & 350 & \\ \hline
      Plate Class V   &   V & 70 & 450 & \\ \hline
      Plate Class VI  &  VI & 80 & 550 & \\ \hline

    \end{tabular}
  \end{center}
\end{table}
