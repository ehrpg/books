%% TODO: Fix formatting.
\newcommand{\skill}[1]{\emph{#1}}

\chapter{Skills}

\subsection{Overview}

When a character peforms complex operations a skill is involved, that your
character has to learn first. Whether it be operating a vehicle, shooting a
firearm, climbing a steep cliff or trying to convince another character or
NPC. Each skill is associated with two abilities. If a skill check is made,
a trained character can add the modifier of those abilities to his roll.

Some skills are special, as in they require a lot of effort to learn and use
right, or require special materials, trainers or facilities to learn properly.
There are also speciality skills. These skills cannot be learned on their own,
but reflect a specialisation of an already existing skill. They help to
specialise and refine a character.

Each skill has a rank that goes from zero to ten. With zero meaning that the
character has just learned the skill and cannot draw upon any experience when
performing tasks. Ten represents absolute mastery.

\subsection{Learning new skills}

Normal skills take one \emph{CCP} to unlock, upon which the skill has a rank
of zero. If the skill is a special skill, the unlock costs are ten \emph{CCP}
instead.

Once the skill is learned, rank increases can be bought with the same exact
number of \emph{CCP} as the next rank.

\subsection{Using Skills}

Each task, against which your skill is pitched against, has a base difficulty.
If the base difficulty is at or below your skill level you may be able to
perform the task without rolling your dice in come circumstances. This means
that your character can perform this task given enough time and no adverse
external influences.

For example, an avid shooter can always hit his target on his range, if he's
rested and aims long enough. But this doesn't mean he can hit another target
on the same distance as easy in the heat of combat, fatigued by days of
fighting. As soon as external influences come into affect, you \emph{have to
  roll}.

If you have unlocked a skill you make a skill check rolling a ten-sided dice,
adding the appropriate ability modifiers, adding the skill rank and adding any
specialised skill ranks (if applicable).

If you have not unlocked the skill you simply roll a ten-sided dice and only
add negative ability modifiers to the roll.

When your result exceeds or is equal to the tasks difficulty (which is the base
difficulty, plus any situational hindrances) you succeed. Please see tasks and
\emph{Checks and Difficulties} for rules on causing minor, medium or major
successes or failures.

\subsection{Specialities}

Specialiaties are specialised sub-fields of another skill. They represent
niches, or specialised area your character can focus on. You are allowed to
take up to three specialised skills (if there are that many), depending on
the rank of the parent skill. Once you reach rank three on the parent skill
you make take one specialised skill, another at rank six and another at rank
nine.

Your specialised skill does not need unlocking, but you cannot have the rank
of the specialised skill above the rank of the base skill. If you perform a
skill check that involves that specialised skill you can add the rank of the
specialised skill to the skill check (alongside the skill rank of the base
skill).

For example, Hana a skilled physician wishes to specialise in Surgery. After
reaching Medicine at rank three, she buys two ranks of Surgery. Now when she
has to perform a surgery she can add the three from Medicine, and the two from
Surgery to the skill check for a total of plus five.

\subsection{Available Skills}

Below you will find a list of all the base skills available in game. It is of
course up the GM to adapt or add to the list below.

\skill{Acrobatics}: With acrobatics you can perform such acts as rope walking,
tumbling or other specialised maneuvers. Uses \emph{Dexterity} and
\emph{Constitution}.

\skill{Acting}: The ability to play a role and successfully emulate behaviours,
emotions and character traits. Uses \emph{Charisma} and \emph{Intelligence}.

\skill{Climbing}: Allows you to climb walls, cliffs and other obstacles. It helps
with finding the right footing while doing so. Uses \emph{Strength} and
\emph{Constitution}.

\skill{Dodging}: A basic combat skill to move out of harm's way. It can be used
as an action to avoid getting hit in melee combat, as well as for ``hitting the
deck'' in case of incoming enemy fire. Uses \emph{Dexterity} and
\emph{Perception}.

\skill{Driving}: Driving involves operating and steering land vehicles (such as
cars, trucks or motorcycles). Driving also includes heavy machinery such as
skid loaders and tracked vehicles (tanks). To fire a tank however, please see
the \emph{Heavy Weapons} skill. It uses \emph{Perception} and \emph{Dexterity}.

\skill{Escape Artist}: This is the skill to escape from ropes, handcuffs and
other restraints. It uses \emph{Strength} and \emph{Dexterity}.

\skill{First Aid}: First aid comprises techniques ranging from applying intrecate
bandages to wounds, performing CPR and Heimlich maneuvers, to some basic skills
and techniques from medicine such as stitching a wound. It uses
\emph{Intelligence} and \emph{Perception}.

\skill{Heavy Weapons}: Heavy weapons are all weapons that cannot be effectively
carried into combat, and include things such as a howitzers, large mortars,
turrets or firing a tank. It uses \emph{Dexterity} and \emph{Perception}.

\skill{Intimidate}: Intimidate allows you to strike fear into the heart of
others. Uses \emph{Strength} and \emph{Charisma}.

\skill{Intrusion}: This skill covers everything from breaking and entering, be
it legally or illegaly. This includes both mechanical and electronic locks and
counter measures (such as alarms, biometric access controls and so forth). It
uses \emph{Dexterity} and \emph{Intelligence}.

\skill{Language}: Each character is assumed to automatically have his mother
tongue at rank 5 (spoken, reading and writing). Any additional language is a
separate skill. Each language skill is based on \emph{Perception} and
\emph{Intelligence}.

\skill{Light Weapons}: This skill grants training in military grade hardware,
such as recoilless rifles, grenade launchers, mortars, light and heavy machine
guns. It uses \emph{Perception} and \emph{Dexterity}.

\skill{Mechanics}: Mechanics allows you to pop open a hood of a car and instantly
recognise why it won't start. It involves repairs and maintenance for machines
big and small, as well as construction of new devices from basic materials. It
requires \emph{Dexterity} and \emph{Intelligence}.

\skill{Medicine (Special)}: Medicine is the science of diagnosing, treating and
preventing diseases and trauma. It also covers therapy and care for patients.
A full degree and licence as a doctor requires years of learning and training.
It has many specialised sub fields such as surgery or trauma care. To treat a
patient use \emph{Perception} and \emph{Intelligence} to diagnose and treat,
and when performing surgery use \emph{Perception} and \emph{Dexterity}.

\skill{Medicine (Bionics) (Speciality)}: This skill allows you to install, repair
and maintain installed bionics.

\skill{Medicine (Surgery) (Speciality)}: Good surgeons perform complicated and
intricate surgeries that sometimes take several hours to complete.

\emph{Medicine (Trauma Care) (Speciality)}: Trauma care specialises in treating
heavy trauma (such as from gun wounds, sharp or blunt weapons or accidents).

\skill{Melee Combat}: You are trained in the art of hand to hand combat. This
skill includes basic training with common melee weapons (knife, batons or
clubs). To hit an opponent you use \emph{Dexterity} and \emph{Strength}.
The damage depends on the weapon used.

\skill{Micro Gravity}: Allows maneuvering in micro gravity environments. Uses
\emph{Dexterity} and \emph{Strength}.

\skill{Navigation}: Allows a character to use charts, maps or a navigational
computer to effectively nagivate from point A to point B. Uses
\emph{Intelligence} and \emph{Perception}.

\skill{Notice}: Notice is used to detect others who are hiding, or to notice
something that is hidden or not obvious. Uses \emph{Perception} and
\emph{Constitution}.

\skill{Pilot Spacecraft (Special)}: This special skill allows you to pilote all
kinds of space vessels. The skill includes things as docking, evasive maneuvers
and dog fighting. It requires \emph{Dexterity} and \emph{Perception}

\skill{Pilot Spacecraft (Dog Fights) (Speciality)}: You know how to respond to
enemy tactics, and can anticipate their next action. This skill represents the
art of close quarter combat in space.

\skill{Pilot Spacecraft (Boarding) (Speciality)}: Disabling, locking down, and
entering enemy space craft for a take over is your speciality. When you use
boarding the abilities change to \emph{Intelligence} and \emph{Perception}.

\skill{Pilot Spacecraft (Smuggling) (Speciality)}: Reducing energy signatures,
hiding and squeezing cargo, and talking your way into and out of deals are the
trademarks of a good Smuggler. When you smuggle goods, the abilities change to
\emph{Charisma} and \emph{Intelligence}.

\skill{Public Affairs}: Knowing who hates and likes who, and how to behave in
various social settings and social events. It also incorporates knowledge about
famous figures, such as politicians, important or influential business people
and so forth. Uses \emph{Intelligence} and \emph{Charisma}.

\skill{Questioning}: The opposite of smooth talking. The art of interrogating,
spotting lies and otherwise get the truth out of people. It uses \emph{Charisma}
and \emph{Intelligence}.

\skill{Running}: This skill includes running for endurance (marathon) as well as
sprinting. It uses \emph{Strength} and \emph{Constitution}.

\skill{Science (Any) (Special)}: You are educated and trained one of the numerous
fields of science. All sciences use \emph{Intelligence} twice, except when
otherwise noted. Some specialities include: \emph{Computer} which incorporates
the science of IT, including programming or hacking; \emph{Physics},
\emph{Chemistry}, \emph{Biology} and various others. Each field is a separate
skill.

\skill{Sleight of Hand}: The art of performing small actions with no one else
noticing. For example cheating at card games, pick pocketing another person
or hide a small weapon on one's own body. It uses \emph{Dexterity} and
\emph{Charisma}.

\skill{Small Arms}: The small arms skill allows a character to aim, shoot, reload
and maintain certain fire arms. Many firearms that are can be learned without
special training, while some others (especially military grade hardware)
requires special training (see \emph{Light Weapons}). Firearms rely on
\emph{Perception} and \emph{Dexterity}. Rifles and handguns handle differently
and are thus split up in two separate speciality skills.

\skill{Small Arms (Handguns) (Speciality)}: Handguns include firearms such as
pistols, revolvers and some specific sub machine guns. They have limited range
and stoping power but are easily concealed.

\skill{Small Arms (Rifles) (Speciality)}: This category includes sporting and
hunting rifles, as well as shotguns.

\skill{Smooth Talking}: The art of lying convincingly, to confuse and deceive
another person. Highly valued by con men and other shady people. Uses
\emph{Intelligence} to fabricate a believable lie, and \emph{Charisma} for
delivering it.

\skill{Stealth}: The art of sneaking, hiding and remaining hidden. Uses
\emph{Dexterity} and \emph{Intelligence}.

\skill{Swimming}: Zero rank allows the character to float, while the first rank
allows a character to swim. Higher ranks allows the character to swim with
clothes, gear or even while helping another who can't swim. Uses \emph{Strength}
and \emph{Constitution}.

\skill{Throwing}: Need to make a grenade go really far away towards the enemy?
Use \emph{Strength} and \emph{Dexterity} to achieve that! This skill also
includes thrown weapons such as throwing axes or knives.
