\chapter{Social Interactions}

In this chapter we will deal with how social interactions between the player
characters and non-player characters (NPCs) should be handled by the game
master.

\section{Interactions with NPCs}

Social interactions range from bartering to convince a trader to pay more
money for a piece of junk than he normally would, from grand galas with many
single interactions between various important figures from business and
politics.

The rule of thumb is that if the conversation is something that influences the
over all story arc, or if the players wish to achieve something particular
with it, a roll should me made for each separate instance. A short speech
might just be a single check, whereas long and important speeches could be
split into multiple checks.

A guide on how to set the DC can be found further down.

\subsection{Player vs. Character}

Players might be great at social interactions whereas the character might not
be and vice versa. If the player delivers a good motivational speech but his
character has low charisma rank and no skills in any speech related skill, it
is advisable to roll anyhow. The same is true the other way around, for the
character's ability are not the same as the player's abilities.

If the player puts extra effort into his speech consider adding a bonus to
the DC, or (in case of positive effects on a roll of \emph{ten}) add something
special or out of the norm.

\subsection{Interacting}

Social interactions come in many forms, shapes and sizes. Trying to convince
someone else that your own viewpoint is right, is one example. A player can
achieve this in many different ways: He can use deceit, intimidation or by
simply trying to talk and arguing the point.

The player decides what he wishes to do and the \emph{GM} decides the DC. For
each instance of interaction a separate roll is needed. If more than one roll
is involved it is common that a few among them might fail. Partial successes
are plausible, for example you might intimidate an engineer into disabling the
security system and lie to him about the reason. Although you might succeed
and he disables the system, he might not believe the reason for why he has to
do so.

To lie and intimidate to an NPC pitch the NPC's \emph{Questioning} skill
against the player's \emph{Oratory} skill roll. Add morale bonus or penalties
(see below) and add any circumstance bonuses depending on the situation. For
example an outrageous lie without a context might incur a higher penalty than
a more believable lie. Or an already scared and tortured captive might be more
susceptible to further intimidation attempts and thus receives a penalty on
his check. The \emph{GM} should define this bonus or penalty depending on the
situation. These penalties or bonuses should range between \emph{minus five}
(-5) and \emph{plus five} (+5) depending on how believable the lie or
intimidation is.

\section{Morale}

The main drive of every person is motivation and morale. People who are
motivated might go the extra mile to achieve something, or if morale is low
might even abandon their current task or goal. Likewise a threat of impending
danger or even death can cause men to achieve the seemingly impossible.

The game represents morale and motivation as a single game statistic for each
character that is added to all roles the players make. It can be negative
(representing loss of morale or motivation), zero (no strong feelings one way
or the other) or positive (motivation to succeed). The morale bonus is
controlled by the \emph{game master} and can be determined from many
different sources.

Values should range between \emph{minus ten} (-10) and \emph{plus ten} (+10),
with minus ten being despair to the point of abandoning one's own believes or
principles, and plus ten being an almost religious frenzy to achieve the
goal, even if it can result in personal harm. The morale bonus has nothing to
do with personal feelings (like fear or anger) although they might stem from
them.

A soldier who has abandoned all hope of returning alive (say a -3 morale
bonus) from this deployment might still charge headlong into battle. Another
example is a crew that has just realised they are stranded in deep space
with no hope for rescue and resources left for just a few hours. The
\emph{GM} decrees that a -4 penalty is appropriate for the situation. After
repairing the emergency distress signal they are contacted by a rescue party
and the \emph{GM} decides that the morale has been restored and is now at +1.

\subsection{Influencing Morale}

Morale can be influenced by successful or failed social checks. For example
a successful Oratory check can increase or restore morale. The skill check for
Oratory is made against \emph{five} plus the morale bonus or the positive
penalty, adding any circumstance penalties as the \emph{GM} sees fit. Such a
speech can made in front of a group of people or towards an individual in
particular. Circumstance penalties or bonuses might stem from the relationship
between the two people, for example a child can be more easily smoothed by his

For example, the crew mentioned above, is marooned in deep space with no help
of rescue. The \emph{GM} the DM sets the morale for everyone to -4, except for
the captain who he set to -2 since this has already happened to him before,
and he has already formulated a plan. The captain addresses a speech at his
crew detailing the plan and trying to boost morale. The DC is as following:

\[
DC = 5 \ub{+ 3}{morale} = 8
\]

The captain has \emph{Oratory} at rank 5.

\[
1d10 = 6 \ub{+ 5}{skill} \ob{- 2}{morale} = 9
\]

He succeeds and the \emph{DM} degrees that the morale is lessened to -2 for
the crew.
