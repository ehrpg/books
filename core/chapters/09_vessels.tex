\chapter{Vessels}
\label{chap:Vessels}

\section{Hitpoints}
\label{sec:Hitpoints}

The hitpoints of anyvessel are composed by its three pools:
\begin{description}
  \item[Armour] Additional Armour protect the vessel from extensive damage.
  \item[Hull] Any damage is applied to the Hull. If the Hull is destroyed, further damage applies to the Structure.
  \item[Structure] The integrity of a vessel is abstracted by Structure Points. If they reach 0, the vessel is completely destroyed.
\end{description}

\section{Hull}
\label{sec:Hull}

The Hull defines the size of a vessel. The size is measured by displacement tons. The size defines how durable a vessel is, but also the the maximum load. Every 50 tons rounded down adds 1 Hull Point and 1 Structure Point, but every vessel has a minimum of 1 Structure Point. Normal vessels start at a size of 100 tons.

Smaller vessels, namely anything with less than 100 tons, also have additional pool values for personal combat. Every 10 tons add 5 Hull Points and Structure Points, to a maximum of 50 Structure Points and 50 Hull Points at a 100 displacement tons.
A small vessel gets an additional Hull Point every 50 tons, and an additional Structure Point at 80 tons.

\begin{table}[H]
  \centering
  \caption{Small Vessels}
  \begin{tabular}{|l|l|l|l|l|}
    \hline
    Tons & \multicolumn{2}{l|}{Personal Scale} & \multicolumn{2}{l|}{Vessel Scale} \\ \hline
    ~    & Hull           & Structure & Hull         & Structure \\ \hline
    10   & 5              & 5         & 0            & 1         \\ \hline
    20   & 10             & 10        & 0            & 1         \\ \hline
    30   & 15             & 15        & 0            & 1         \\ \hline
    40   & 20             & 20        & 0            & 1         \\ \hline
    50   & 25             & 25        & 1            & 1         \\ \hline
    60   & 30             & 30        & 1            & 1         \\ \hline
    70   & 35             & 35        & 1            & 1         \\ \hline
    80   & 40             & 40        & 1            & 2         \\ \hline
    90   & 45             & 45        & 1            & 2         \\ \hline
    100  & 50             & 50        & 2            & 2         \\ \hline
  \end{tabular}
\end{table}

\section{Components}
\label{sec:Components}

A vessel is composed by multiple components. These are interchangeable, as long as there is enough space in the vessel. While some components are required for a vessel to function properly, others are optional.

The components can be categorized into:
\begin{description}
  \item[Armour]
  They protect the ship against enemy weapon fire or hazardous environments.
  \item[Rooms]
  Define the layout and functionality of the interior of a (larger) vessel.
  \item[Subsystems]
  Contains various technologies that are used on a vessel. This includes jump drives, reactors, scanners, etc.
  \item[Weapons]
  Anything from simple cannons to complex missile launchers or railguns are found in this category.
\end{description}

\subsection{Power plant}

A vessel can be supplied by power by a fusion reactor or the newer and much more expensive matter/anti-matter (M/AM) reactor.

\subsection{Fuel}
\label{sub:Fuel}

Fusion reactors are fueled by processed hydron fuel cells, while  a matter/anti-matter (M/AM) reactor uses matter/anti-matter cells as fuel.

\subsection{Propulsion}


\subsection{Jump drives}

Jump drives allow interstellar and even intergalactic travel but require a lot of power to be initiated.

\subsubsection{Warp drive}\footnote{Also known as Alcubierre drive.}

Requires a the energy output of a fusion reactor.

\subsubsection{Hyperdrive}\footnote{Also known as Einstein-Rosen-Bridge drive.}

Requires a the energy output of a matter/anti-matter reactor.

\section{Fire Arcs}
\label{sec:Fire Arcs}

The space around a vessel is divided into 4 90 degree arcs in a two dimensional space. In three dimensions the arcs have an additional 90 degree span up and down respectively. The arcs are labelled clockwise, starting in the front:

\begin{itemize}
  \item Fore Arc
  \item Starboard Arc
  \item Aft Arc
  \item Port Arc
\end{itemize}

\section{Hardpoints \& Turrets}
\label{sec:Hardpoints & Turrets}

Any vessel has hardpoints whereas each serves as a platform for exactly one turret. The turret defines how many weapons can fit onto it. These weapons are always of the same type or else the fire control wouldn't function properly. A hardpoint can be located anywhere on the ship, but usually the chassis dictates where a weapon mount is located.

\subsection{Bearing}
\label{sub:Bearing}

The bearing of a turret defines in which arcs a battery can fire at. Usually, either the front or the side arcs have the main weaponry, while the back is only sparsely outfitted. A spaceship f.e. may have fighter or bomber defense in the rear, but no larger weaponry.

\section{Travel}
\label{sec:Travel}

\subsection{Sublight Travel}
\label{sub:Sublight Travel}

Sublight propulsion is cheaper and more cost efficient than other drives. This component defines how long it takes to fly from origin to destination without hyperspace, but also how much Speed Points a vessel has in combat.

\subsection{Hyperspace}
\label{sub:Hyperspace}

A jumpdrive uses energy once to bend the space. The energy used depends on the distance from origin to destination and the mass of the vessel. Using a jump drive creates a radioactive isotope. The radioactivity is not harmful to humans, but it can be very easily picked up by modern radars. Analysis of the radioactive residue allows an approximation about when a vessel activated its jump drive and how big the vessel was.

\section{Armour}
\label{sec:Armour}

\begin{table}[H]
  \centering
  \caption{Armour}
  \label{tab:vessel-armour}
  \begin{tabular}{|l|l|l|l|}
    \hline
    Armour                    & Protection & Cost         & Weight      \\  \hline
    Concrete                  & 1          & 5\% of hull  & 5\% of hull \\ \hline
    Steel platings            & 2          & 10\% of hull & 5\% of hull \\ \hline
    Tungsten carbide platings & 4          & 30\% of hull & 5\% of hull \\ \hline
    Depleted Uranium platings & 5          & 50\% of hull & 5\% of hull \\ \hline
  \end{tabular}
\end{table}

\section{Weapons}
\label{sec:Weapons}

\begin{table}[H]
  \centering
  \caption{Weapons}
  \label{tab:vessel-weapons}
  \begin{tabularx}{\textwidth}{|l|l|l|l|l|X|}
    \hline
    Weapon       & Damage      & Range       & Cost & Weight  & Notes                            \\ \hline
    Cannon       & 2d6         & 50 km       & ~    & 5 tons  & ~                                \\ \hline
    FLAK Cannon  & 1d4         & 5 km        & ~    & 2 ton   & Requires 2 Speed Points to Dodge \\ \hline
    Gatling Gun  & 1d8         & 20 km       & ~    & 2 ton   & ~      \\ \hline
    MRL          & See Missile & See Missile & ~    & 20 tons & Can fire up to 12 missiles       \\ \hline
    Torpedo tube & See Torpedo & See Torpedo & ~    & 50 tons & Can only make 1 attack per turn  \\ \hline
    Rail Gun     & 2d12     & 500 km      & ~    & 40 tons & Ignores Armour                   \\ \hline
  \end{tabularx}
\end{table}

\begin{table}
  \centering
  \caption{Torpedos}
  \begin{tabular}{|l|l|l|}
    \hline
    Torpedo            & Damage & Notes                                               \\ \hline
    Gravity Torpedo    & 2d6   & Reduces Speed Points by 1.                          \\ \hline
    Nuclear Torpedo    & 2d8   & Applies lethal radiation to location.               \\ \hline
    Antimatter Torpedo & 2d12   & On hit also effects directly surrounding locations. \\ \hline
  \end{tabular}
\end{table}

\begin{table}
  \centering
  \caption{Missiles}
  \begin{tabularx}{\textwidth}{|l|l|l|l|X|}
    \hline
    Missile         & Damage & Cost & Weight   & Notes                                                      \\ \hline
    Basic Missile   & 1d6    & 1000 & 0.1 tons & ~                                                          \\ \hline
    Guided Missile  & 1d6    & 2500 & 0.1 tons & Rolls an attack every turn until it hits or it's destroyed \\ \hline
    Nuclear Warhead & 2d6    & 5000 & 0.5 tons & Applies lethal radiation to location                       \\ \hline
  \end{tabularx}
\end{table}

\chapter{Vessel Combat}
\label{chap:Vessel Combat}

The combat with vessels is similiar to the rules described in Chapter~\ref{chap:Combat}, unless otherwise noted.

\begin{enumerate}
  \item Setup
  \begin{enumerate}
    \item Determine range between ships and their relative position.
    \item Determine crew positions.
    \item Determine initiative.
  \end{enumerate}
  \item Combat Phase
  \begin{enumerate}
    \item Act
    \item React
    \item Resolve
  \end{enumerate}
\end{enumerate}

\subsection{Range \& Position}
\label{sub:Vessels-Combat-Setup-Range}

The distances are similiar to those in Chapter~\ref{chap:Combat}:

\begin{table}[H]
  \centering
  \caption{Range}
  \label{tab:vessel-range}
  \begin{tabular}{|l|l|}
    \hline
    Range    & Distance       \\ \hline
    Adjacent & < 1km          \\ \hline
    Close    & 1km - 10km     \\ \hline
    Medium   & 10km - 100km   \\ \hline
    Far      & 100km - 1000km \\ \hline
    Distant  & > 1000km       \\ \hline
  \end{tabular}
\end{table}


Also, before combat begins be sure to determine the relative positions of each ship, and which side is facing where. Since it takes Speed Points to turn in combat and turrets can not fire into all arcs, the orientation is important too.

\subsection{Crew Positions}
\label{sub:Vessels-Combat-Setup-Crew-Positions}

If the vessel is larger so it requires more people to effectively use it in combat, the following positions become available. Some of them can only be held by one person, while others, such as the gunner or engineer, can be held by multiple people.

\begin{description}
  \item[Captain] Orders the crew, determines the tactic used, etc.
  \item[Pilot] The pilot maneuvers the vessel.
  \item[Engineer] An engineer is responsible for repairs.
  \item[Gunner] A gunner uses any of the vessels turret.
  \item[Marine] The marine is a combat trained unit, trained to board other ships or defend the ship from boarding parties.
  \item[Passenger] A passenger typically is not involved in the combat in any way.
\end{description}

\subsection{Artificial Intelligence}
\label{sub:Vessels-Combat-Setup-Artificial-Intelligence}

Some of the position listed in~\ref{sub:Vessels-Combat-Setup-Crew-Positions} can be assumed by artifial intelligence components. These components are used for combat. Possible AI combat modules include:

\begin{description}
  \item[Piloting and navigation module]
  Automatically dodges oncoming fire instead of the pilot.
  \item[Fire-control]
  Any fire-control AI can assume control of all turrets on a vessel.
  \item[Damage mitigation module] This module seals off exposed areas, ventilates air in case of fire, etc. to prevent crew damage.
\end{description}

\subsection{Initiative}
\label{sub:Vessels-Combat-Setup-Initiative}

The captain of a vessel uses his \emph{Tactics} skill to determine the initiative. If a vessel has no captain, use the pilots \emph{Pilot (any)} skill instead.

\section{Combat Phase}
\label{sec:Vessels-Combat-Combat-Phase}

Combat takes place in turns. A player may make two actions per turn and use an indefinite amount of reactions on the enemies turn. While normal combat is handled each player at a time, the crew of a vessel coordinate their actions and execute them simultaneously. Also, turns last much longer than in normal combat, up to several minutes.

\subsection{Actions}
\label{sub:Actions}

Some valid actions are:

\begin{description}
  \item[Attack]
  Each gunner and AI may fire all weapons at one ore more enemies. If the enemy has no point defense and does not dodge, the vessel is hit. To deliberately attack a certain subsystem of the enemy vessel, the attacker incurs a -5 penalty.
  \item[Damage]
  If a vessel is hit, roll damage for the weapon. Substract the vessels armour from the roll. The result determines how often you hit the enemy with an attack, see Table~\ref{tab:vessel-damage}. For each hit roll for the location hit, see Table~\ref{tab:vessel-location}.
  \item[Move]
  A vessel can either move by 1 square its facing or turn 90 degrees in any direction per Speed Point. If no more Speed Points are available the vessel cannot be moved anymore until the next turn.
  \item[Repair]
  In combat subsystems can be repaired by making a successful \emph{Engineering} skill check. The difficulty to repair a component is defined by component itself. If you succeed, one hit is repaired. For each 3 points you exceed the DC you can repair an additional hit, up to 3 hits per skill check.
  \item[Boarding]
  If your vessel has a boarding crew and the vessels in question are adjacent, the pilot may make a \emph{Pilot (any)} skill checked opposed by the enemies pilot's piloting skill check. If you succeed the enemy vessel is boarded, then switch to normal combat rules. If you fail, the boarding attempt fails.
\end{description}

\begin{table}[H]
  \centering
  \caption{Damage}
  \label{tab:vessel-damage}
  \begin{tabular}{|l|l|}
    \hline
    \textbf{Damage}              & \textbf{Effect}                    \\ \hline
    0 or less                    & No damage                          \\ \hline
    1-4                          & Single Hit                         \\ \hline
    5-8                          & Two Single Hits                    \\ \hline
    9-12                         & Double Hit                         \\ \hline
    13-16                        & Three Single Hits                  \\ \hline
    17-20                        & Two Single Hits, Double Hit        \\ \hline
    21-24                        & Two Double Hits                    \\ \hline
    25-28                        & Triple Hit                         \\ \hline
    29-32                        & Triple Hit, Single Hit             \\ \hline
    33-36                        & Triple Hit, Double Hit             \\ \hline
    37-40                        & Triple Hit, Double Hit, Single Hit \\ \hline
    41-44                        & Two Triple Hits                    \\ \hline
    For every extra three points & +1 Single Hit                      \\ \hline
    For every extra six points   & +1 Double Hit                      \\ \hline
  \end{tabular}
\end{table}

\begin{table}[H]
  \centering
  \caption{Location}
  \label{tab:vessel-location}
  \begin{tabular}{|l|l|l|l|}
    \hline
    \textbf{2d6} & \textbf{Small Craft} & \textbf{External Damage} & \textbf{Internal Damage} \\ \hline
    2   & Hull                & Hull                & Crew            \\ \hline
    3   & Fusion reactor      & Sensors             & Jump Drive      \\ \hline
    4   & Sublight Propulsion & Sublight Propulsion & Fusion reactor  \\ \hline
    5   & Fuel                & Fuel                & Bay             \\ \hline
    6   & Hull                & Hull                & Structure       \\ \hline
    7   & Armour              & Armour              & Cargo           \\ \hline
    8   & Hull                & Hull                & Structure       \\ \hline
    9   & Turret              & Turret              & Bay             \\ \hline
    10  & Hold                & Sublight Propulsion & Jump Drive      \\ \hline
    11  & Fusion reactor      & Sensors             & Fusion reactor  \\ \hline
    12  & Bridge              & Hull                & Bridge          \\ \hline
  \end{tabular}
\end{table}

\subsection{Reactions}
\label{sub:Reactions}

\begin{description}
  \item[Dodging]
  Reactions are mainly used for dodging oncoming fire. This is only possible if there are Speed Points available to do so. Each dodging maneuver generally costs 1 Speed Point, but to dodge some weapons more Speed Points are required. The pilot then makes a \emph{Pilot (any)} check against the attackers roll. If he beats the attack, he dodges the oncoming fire, otherwise the vessel is hit normally.
  \item[Point Defense]
  The gunner or the AI may attack oncoming missiles and torpedos rolling against the attack. Each subsequent point defense roll applies a cumulative penalty of 1. If the roll beats the attack, the missile or torpedo is destroyed, otherwise vessel is hit normally.
\end{description}
