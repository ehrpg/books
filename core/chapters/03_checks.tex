\chapter{Checks and Difficulties}

\subsection{Overview}

Everything in this game is done by making a \emph{check} against a given
\emph{difficulty}. It is the players job to make the check, while it is the
game masters job to keep track of the difficulty. When bonuses or penalties
are given in this document, they apply to the difficulty \emph{and not the
  check}.

This is important as the player probably does not, and should not, know all
the factors that make his work harder. As a game master you should always be
be as transparent as possible when it comes to determining the difficulty.

The game does not give the GM a detailed list of specific bonuses or penalties,
and it is up the GM's discretion to specify this. As an example, a fog can
add from +1 to +6 to the difficulty of a task to identify another person,
depending on how thick the fog is.

\subsection{Rolling}

Almost all of the skill checks in \emph{Endless Horizons} are done by rolling
a ten sided die (d10). The player adds any relevent bonus from his or her
character, while the game master calculates the difficulty. If the player's
result is at or above the difficulty he succeeds.

If the player rolls a one (1) or a ten (10) he has caused the dice to escalate.
The player rolls again until either he rolls a number that is not one or ten,
or until he has rolled three times in total. Rolling a ten gives a minor
positive effect, rolling a one adds minor negative effect. Rolling two tens or
two ones in a row adds a medium positive or negative effect; and rolling three
ones or tens in a row adds a major positive or negative effect to the outcome
off the task.

Positive and negative effects cancel each other out. So if the player rolls a
one, and then rolls a ten; these two effects cancel each other. If the player
rolls a ten, then another ten and finally a one, the result is just minor
positive effect as one ``ten'' and the ``one'' cancel each other.

A check is not automatically failed when rolling a one. A check only fails if
the result does not meet the difficulty.

\subsection{Effects}

When the player achieves a minor, medium or major effect it is up the game
master to decide what happens. These effects happen regardless of whether the
check succeeds or not.

Minor positive effects grant a small immediate bonus (for example extra
damage, or the task is done faster), while minor negative effects do the
opposite. For example a task could take longer to complete, or the player
does less damage than normal.

Medium effects give major bonuses (for example they might take down the target
outright) or major disadvantages, for example the weapon jams and needs to be
cleared and reloaded before it is usable again.

Major effects are relatively rare, but may bring devastating results not only
for the player in question but maybe also for others around him, or allow him
to achieve the seemingly impossible.

Example: The player wishes to shoot a small burst with his rifle against a
target down the hallway. He rolls a 1, but after all the bonuses are added
the GM determines that he hits. The player rolls again (a 6) and breaks the
escalating die. He hits, although with a minor negative effect. The game
master decided that in the heat of battle the player overshot, and spent
more bullets than he intended. The player shot a burst of ten instead of five,
yet none of the extra bullets hit any enemies.
