\chapter{Vessels}
\label{chap:Vessels}

% Spacecrafts only for now.


\section{Definition}
\label{sec:Vessels-Definition}

A vessel is a highly customizable craft that fulfills a purpose, usually things like transport, expedition, scouting, security or war. The chassis is the foundation for any vessel and is essentially a frame that outlines the shape, amount of slots, etc.

\section{Components}
\label{sec:Vessels-Components}

Components define a vessel and are subdivided into the following four categories:

\begin{description}
  \item[Armor] Armor platings can be used to further enhance a ships durability in battle or hazardous environment.
  \item[Rooms] Rooms are only found on larger vessels. This category includes rooms like the crew quarter, medic bay, cargo bay, engine room, bridge, etc.
  \item[Subsystems] This category is used for everything, that doesn't fit in any of the other categories. They include shield generators, warp drives, engines, fuel tanks, but also components like the cargo hatch can be found in this category.
  \item[Weapons] Weapons can range from simple cannons to complex missile launchers, railguns, etc. Also, ammo is included in this category.
\end{description}

A component and it's purpose are defined by multiple stats. These stats can differ from component to component, based on it's use, but every component includes the following stats:

\begin{description}
  \item[Size] The size tells you the minimum size of the slot the vessel requires. You cannot, f.e., add a medium component to a small ship.
  \item[Slots] The amount of slots that are needed to install the component needs.
  \item[Hitpoints] Every component's condition is descriped by it's hitpoints. If a component is damaged, so that it doesn't have it maximum hitpoints anymore, it's usually disabled, temporarily destroyed, but it can still be repaired. If its hitpoints are brought to 0, the component counts as destroyed and cannot be made operational again. Some components may have secondary effects when destroyed. F.e. if the ammo dump of a vessel is destroyed, as in brought to 0 hitpoints, there is a chance that the remaining ammo will explode which would not only deplet the enemies ammo, but also cause additional damage to the vessel.
\end{description}


\section{Identifying Vessels}
\label{sec:Vessels-Identifying Vessels}

Any vessel can be uniquely assembled, although most ships are created in mass production. In the latter case, it's very easy to identify which components a vessel has or not. Otherwise, scanners, espionage and the like might shed some light on the exact vessel composition. Components, such as \emph{rooms}, can only be identified when boarding or hacking the vessel.

\section{Combat}
\label{sec:Vessels-Combat}

The combat in space with vessely is very similiar to the rules described in Chapter~\ref{chap:Combat}, unless otherwise noted.

The attacker can choose to attack the vessel directly or aim to destroy or disable any component the vessel may have. The vessel itself is usually way more durable than any components, so attacking certain components might prove very useful.

Any pilot of a highly maneuverable vessel can use his \emph{Dogfighting} skill to try and evade an enemy attack before it happens. Likewise, the attacker can use the exact some skill to determine, if he outmaneuvers his target and is thusly able to get a clear shot. If the attacker he success, he attackes as usual, if he fails the contest, he cannot shot at his enemy.

Any vessel that is to bulky, large or sluggish to actively evade an oncoming attacking can be attacked directly as usual.

% TODO
DRAFT: In the case that a high maneuverable vessel attacks a large, sluggish one, the pilot of the high maneuverable vessel can add his  piloting skill to the DC the attacker has to beat when being attacked.
