\chapter{Injury and Death}

\section{Hit Points}
\label{sec:hp}

Your hit points determine how much kick you have left in you. If you drop to or
below zero you die. No matter how little hit points you have left your character
is not hindered in any way. Wounds however, can hinder your character and must
be treated.

\subsection{Recovering Hit Points}
\label{sub:recoverhp}

If all wounds are cared for, and you have time to rest and heal, you roll to
see if you naturally recover one hit point per week. Any straneous activity -
such as fighting or traveling - interferes with this natural healing ability,
and no hit point is recovered in that week. To see if you can recover one hit
point make a consitution check against a DC totalling the amount of wounds you
have (see below for wounds).

Someone who is trained in \emph{Medicine} can speed up this healing process.
First roll to \emph{diagnose} the patient and then to \emph{treat} the wounds.
The DC is as followed:

\[
DC = 5 + \frac{\ob{hp}{HP lost}}{2}
\]

If the diagnose check is successful any difference can be added to the treatment
roll as a bonus. If the diagnose check was unsuccessful any difference is
substracted from the treatment roll as a penalty.

Then roll for treatment, with the same DC, and add or subtract any relevant
bonus or penalty from the diagnose check. If you succeed you can heal one hit
point per day for one week. If you fail however, you do not cure any additional
hit points. Diagnosis and treatment requires a medic lab and at least a few
hours of work, and thus cannot be done again on the same day.

A character is who is trained in \emph{first aid} can do the same, but the DC is
is more difficult:

\[
DC = 5 + \ub{hp}{HP lost}
\]

\section{Wounds}
\label{sub:Wounds}

The wounding system abstracts away injury, wounds and other impediments a
character gains through the loss of hit points. Having suffered ``four wound
damage'' does not mean, that the character has four minor wounds. They might as
well be one major wound that has gradually gotten worse.

Whenever a player takes damage (from any source) he has to roll to see whether
he sustains a wound. Wounds are separate from HP loss and represent negative
effects a player might sustain for losing HP.

To see if the player gains a wound from roll a constitution check against the
following the damage taken. This is the damage after any relevant armour is
subtracted and before any zone modifiers are applied.

If the check fails you gain a wound. This wound reduces your maximum hit points
by one until it is treated. This effect stacks with all other wounds, and may
even kill you if your maximium hit points reaches zero. The hit point loss from
reducing your maximium hit points, does in itself not cause another wound.

\subsection{Treating Wounds}

A character capable \emph{Medicine} or \emph{First Aid} can try to treat a
wound with a \emph{treatment} check. The DC is the amount of wounds (or the
difference between your current and actual maximum HP) you have suffered. If
\emph{Medicine} is used to treat the wound, you can attempt a \emph{diagnose}
check to see if you gain a bonus to your treatment dice (just as described in
\emph{Recoverig Hit Points}). If the check succeeds one wound is cured, and
thus one maximum HP is restored.

\[
DC = \ob{wounds}{Wound points suffered}
\]

Wounds cannot be cured naturally over time, and either require specialised
equipment or others to diagnose and treat you. A character can try to diagnose
and treat himself, although he gains a +2 penalty to such checks. The DC is
thus:

\[
DC = \ob{wounds}{Wound points suffered} \ub{+ 2}{penalty for self treatment}
\]

\subsection{Major Wounds}

If your maximum HP is reduced below one half of your normal HPs through wounds
you have suffered a \emph{major wound}. Whenever you do any strenous activity on
a day, you must roll a constitution score against your wound total. If you fail
this check you gain another wound point.

If your maximum HP is reduced below one quarter of your normal HP through wounds
you have suffered a \emph{fatal wound}. For every day your maximum hit points is
reduced below one quarter, you must make a constitution score against the wound
total. If you fail you gain another wound point.

\[
DC = \ob{wounds}{Wound points suffered}
\]

\subsection{Wounding Effects}

Whenever your wound points reach a multiple of five (5, 10, 15 etc.) you can one
additional wounding effect. Roll ten sided dice against the table below to
determine this adverse affect.

If your wound points are reduced below the nearest multiple of five, that
negative effect is cured. If you reach that multiple of five again you
immediately gain a new adverse affect. For example if you have 10 wound points,
and a doctor cures you for one wound points (to 9) you lose whatever affect you
have received when reaching the tenth wound point. But as soon as you take a
wound again (thus reaching 10 again) you have to reroll to see what affect this
has.

\begin{table}
  \caption{Adverse affects due to wounds}
  \begin{center}
    \begin{tabular}{|l|l|}
      \hline
      Result & Effect             \\ \hline
      1      & 2 wounds           \\ \hline
      2      & 1 wound            \\ \hline
      3      & -2 speed           \\ \hline
      4      & -2 strength        \\ \hline
      5      & -2 dexterity       \\ \hline
      6      & -2 constiution     \\ \hline
      7      & -2 perception      \\ \hline
      8      & 3 points of damage \\ \hline
      9      & 1 point of damage  \\ \hline
      10     & -                  \\ \hline
    \end{tabular}
  \end{center}
\end{table}

These effects stack with each other. If you roll a ten nothing happens.
